\documentclass{beamer}
\usepackage[T1]{fontenc}
\usepackage[utf8]{inputenc}
\usepackage{lmodern}
\usepackage[ngerman]{babel}

\usepackage{graphics}

\usepackage{amsmath}
\usepackage{booktabs}
\usepackage{listings}

\usepackage{dot2texi}
\usepackage{tikz}
\usetikzlibrary{arrows,shapes}
 \usetikzlibrary{matrix}
\usetikzlibrary{automata}

\usepackage{wrapfig}

\usetheme{Singapore}
\usecolortheme{dove}
\graphicspath{{images/}{../comics/}}

\newcommand{\xn}{\visible<2->{$\times$}}
\newcommand{\xj}{\visible<2->{$\checkmark$}}
\newcommand{\rd}[1]{\textcolor{red}{#1}}
\newcommand{\gn}[1]{\textcolor{green}{#1}}
\newcommand{\bl}[1]{\textcolor{blue}{#1}}

\newcommand{\hiddencell}[2]{\action<#1->{#2}}

\newcommand{\link}[2][blue]{\underline{\textcolor{#1}{#2}}}
\renewcommand{\emph}[1]{\textit{\textcolor{gray}{#1}}}
\newcommand{\warn}[1]{\textcolor{red}{#1}}
\newcommand{\ans}[2]{\visible<#1->{\textcolor{blue}{#2}}}
\newcommand{\xb}[1]{{\bf #1}}   % \xb {fett}
\newcommand{\xu}{\underline}    % \xu {unterstrichen}
\newcommand{\hide}{\onslide+<+(1)->}

\AtBeginSection[]
{
  \begin{frame}[plain]
    \frametitle{}
    {\footnotesize
      \tableofcontents[currentsection]
    }
  \end{frame}
}

\title{Grundbegriffe der Informatik}
\author{Patrick Niklaus}

\begin{document}
\begin{frame}
  \frametitle{Grundbegriffe der Informatik}
  \framesubtitle{9. Tutorium}
  \begin{description}
    \item \textbf{Name:} Patrick Niklaus
    \item \textbf{E-Mail:} patrick.niklaus@student.kit.edu
    \item \textbf{Nr:} 43
  \end{description}
\end{frame}

\section{Übungsblatt}
\begin{frame}
  \frametitle{Anmerkungen zum letzten Übungsblatt}
  \begin{enumerate}
    \item Bla
  \end{enumerate}
\end{frame}

\section{Endliche Automaten}
\subsection*{}
\begin{frame}
  \frametitle{Arten von Automaten}
  \begin{block}{Mealy-Automat}
    \begin{itemize}
      \item Erzeugung einer Ausgabe bei jedem Zustandsübergang
      \item Ausgabefunktion $g: Z \times X \rightarrow Y^*$
      \item Markieren der \emph{Kanten} mit $x_i|y_i$
    \end{itemize}
   \end{block}
  \pause
  \begin{block}{Moore-Automat}
    \begin{itemize}
      \item Erzeugung einer Ausgabe bei Erreichen eines Zustands
      \item Ausgabefunktion $h: Z \rightarrow Y^*$
      \item Markieren der \emph{Zustände} mit $q_i|y_i$ ($q_i$ ist Zustandsname)
    \end{itemize}
  \end{block}

  In beiden Fällen ist die Ausgabe ein Wort $y = y_0\ldots y_{n-1}$ über einem
  Ausgabealphabet $Y$.
\end{frame}

\begin{frame}
  \frametitle{Endlicher Akzeptor}
  Spezialfall eines Moore-Automaten: Ausgabealphabet $Y = \{0, 1\}$
  \begin{definition}
    $A = (Z, z_0, X, f, F)$
    \begin{description}
      \item[$Z$:] Zustandsmenge
      \item[$z_0$:] Startzustand $z_0 \in Z$
      \item[$X$:] Eingabealphabet
      \item[$f$:] $Z \times X \longrightarrow Z$, Zustandsübergangsfunktion
      \item[$F$:] Akzeptierende Zustände
    \end{description}
  \end{definition}
  \begin{itemize}
    \item Ein Wort $w\in X^*$ wird \emph{akzeptiert}, wenn gilt $f^*(z_0,w)\in F$
    \item Die von einem Akzeptor $A$ \emph{akzeptierte formale Sprache} ist $L(A)=\{w \in X^* |f^*(z_0,w)\in F\}$
   \end{itemize}
\end{frame}


\section{Reguläre Ausdrücke}
\subsection{Definitionen}
\begin{frame}
  \frametitle{Definition}
  \emph{Reguläre Ausdrücke} sind eine verbreitete Notation, um reguläre
  Sprachen (Typ-3) zu beschreiben.
	\begin{block}{Syntax}
		\begin{tabular}{c p{0.7\textwidth}}
				\xb{Zeichen} & 	\xb{Bedeutung} \\
				$( )$			& Klammerung von Alternativen\\
				$\cdot$			& Verkettet Ausdrücke\\
				$|$			& trennt Alternativen\\
				$*$			& beliebig häufiges Vorkommen\\
		\end{tabular}
	\end{block}
  \begin{itemize}
      \item $*$ bindet stärker als Verkettung
      \item Verkettung $(R \cdot S)$ bindet stärker als "`oder"' $(R|S)$
  \end{itemize}
\end{frame}

\begin{frame}
  \frametitle{Die Sprache von R}
	\begin{definition}
		Sei $R$ ein regulärer Ausdruck, dann bezeichnet $\langle R \rangle$ die von ihm erzeugte Sprache.
    Es gilt:
    \begin{itemize}
      \item $\langle \emptyset \rangle =\{\}$
      \item Für $a \in A$ ist $\langle a \rangle=\{a\}$
      \item $\langle R_1 | R_2 \rangle = \langle R_1 \rangle \cup \langle R_2 \rangle$
      \item $\langle R_1 R_2 \rangle = \langle R_1 \rangle \cdot \langle R_2 \rangle$
    \end{itemize}
    $R_1$ und $R_2$ sind hier zwei beliebige reguläre Ausdrücke.
	\end{definition}
\end{frame}

\begin{frame}
  \frametitle{Beispiele}
	\begin{exampleblock}{}
    \begin{itemize}
      \item $R_1 = ({(a|b)*}c)|({(c|b)*}a)$
      \item $R_2 = {(a|b|c|\dots|z)*}.jpg$
      \item $R_3 = {(a|b|c|\dots|z)*}@{(a|b|c|\dots|z)*}.{(a|b|c|\dots|z)*}$
      \item $Z := (a|b|c|\dots|z)$\\
            $R_4 = \text{http://}{Z*}.{Z*}(\emptyset|/Z*)*$
    \end{itemize}
	\end{exampleblock}
\end{frame}

\subsection{Aufgaben}
\begin{frame}
  \frametitle{Aufgaben}
  \begin{enumerate}
		\item Welche Wörter erzeugt der reguläre Ausdruck: $R=(a|b)*abb(a|b)*$\\
    \ans{2}{$\langle R \rangle$ enthält alle Wörter mit dem Teilwort $abb$}
		\item Gebe einen regulären Ausdruck für die Sprache aller Wörter die nicht $ab$ enthalten\\
		\ans{2}{$R = {b*} {a*}$}
  \end{enumerate}
\end{frame}

\begin{frame}
  \frametitle{Aufgaben}
  \begin{enumerate}
		\item Welcher reguläre Ausdruck R erzeugt die Sprache $L=\{\epsilon\}$?
      \ans{2}{$\emptyset*$, denn $\langle \emptyset \rangle ^* = \{\}^* = \{\epsilon\}$}
		\item Gebe einen regulären Ausdruck für die Sprache aller Wörter mit mindestens 3 b's an!\\
			\ans{2}{${(a|b)*}{b(a|b)*}{b(a|b)*}{b(a|b)*}$ oder ${a*}{ba*}{ba*}b{(a|b)*}$}
	\end{enumerate}
\end{frame}

\begin{frame}
  \frametitle{Aufgabe}
  Gegeben ist folgende Klasse von Sprachen über dem Alphabet $\Sigma = \{a,b,c\}$:
  \begin{multline*}
    L_n = \{w|
      \text{w enthält genau einmal das Teilwort $w'=a^n$}\\
      \text{das nicht Teil von $a^k$ mit $k > n$ ist.}
          \}
  \end{multline*}
  Gebt einen regulären Ausdruck $R$ für $L_4$ an! (also: $\langle R \rangle = L_4$)\\
  \hfill
  \ans{2}{
    Wir stellen sicher, dass $aaaa$ genau einmal vorkommt, und sonst nur 1, 2, 3 oder mehr $a$'s.
    \begin{multline*}
      R = ( (b|c)* (\emptyset |a|aa|aaa|aaaaaa*)(b|c)(b|c)*)* \\ aaaa (
      (b|c)(b|c)*(\emptyset |a|aa|aaa|aaaaaa* )(b|c)*)*
    \end{multline*}}
\end{frame}

\begin{frame}
  \frametitle{Aufgabe}
	\begin{exampleblock}{}
		Sei $R$ ein regulärer Ausdruck für eine formale Sprache $L=\langle R \rangle$. Konstruiere einen regulärer Ausdruck
  		\begin{enumerate}
        \item für $L^*$ \\
          \ans{2}{$R_* = (R)*$}
        \item für $L^+$ \\
          \ans{2}{$R_+ = R(R)*$}
  		\end{enumerate}
	\end{exampleblock}
\end{frame}


\section{Rechtslineare Grammatiken}
\subsection{Definition}
\begin{frame}
	\frametitle{rechtslineare Grammatiken}
	\begin{definition}
  		Eine rechtslineare Grammatik ist eine kontextfreie Grammatik $G=(N,T,S,P)$ mit folgenden Einschränkungen:\\
      Alle Produktion sind von der Form:
  		\begin{description}
        \item[] $X \rightarrow w$
        \item[] $X \rightarrow wY$ \hfill ($w \in T^*$\hspace{0.25em} $X,Y \in N$)
  		\end{description}
	\end{definition}
\end{frame}
\begin{frame}
  \frametitle{Äquivalenz}
   Zu jeder rechtslinearen Grammatik gibt es:
	\begin{itemize}
		\item \ldots einen entsprechenden regulären Ausdruck und \pause
		\item \ldots einen einen deterministischen endlichen Automaten
   \end{itemize}
\end{frame}

\subsection{Beispiele}
\begin{frame}
	\frametitle{Rechtslinear?}
	\begin{exampleblock}{Ist diese Grammatik rechtslinear?}
    $G=(\{X,Y\},\{a,b\},X,\{X \rightarrow aY | \epsilon, Y \rightarrow Xb\})$\\
    \begin{description}
      \visible<2->{\item[\emph{nicht} rechtslinear:] Betrachte die Produktion $Y \rightarrow Xb$
      \visible<3->{\item[erzeugte Sprache:] $L(G)=\{a^k b^k | k \in \mathbb N_0 \}$}\\}
    \end{description}
	\end{exampleblock}
  \visible<4->{Kann es eine rechtslineare Grammatik für diese Sprache geben? \\ Ist diese Sprache regulär?}
  \visible<5->{\warn{Nein, ist sie nicht!}}
\end{frame}

\subsection{Aufgaben}
\begin{frame}
\frametitle{Aufgabe}
	\begin{block}{Grammatik $\rightarrow$ Automat}
		$G=(\{X,Y,Z\},\{a,b\},X,P)$\\
    $P=\{X \rightarrow aX | bY | \epsilon, Y \rightarrow aX | bZ | \epsilon, Z \rightarrow aZ | bZ \}$
    \begin{enumerate}
			\item Was ist $L(G)$?
			\item Geben Sie einen endlichen Automaten an, der L(G) akzeptiert.
			\item Lässt sich diese Grammatik noch vereinfachen?
    \end{enumerate}
	\end{block}
	\visible<2->{
	\begin{block}{Lösung}
      		\begin{tikzpicture}[shorten >=1pt,node distance=2cm,auto,initial text=,>=stealth]
        		\node[state,initial,accepting]  (q_0)                       {$0$};
       			\node[state,accepting]          (q_1) [right of= q_0] {$1$};
        		\node[state]                    (q_2) [right of= q_1] {$2$};
        		\path[->] (q_0) edge [loop below]      node        {$a$} ()
        		edge [bend right] node [swap] {$b$} (q_1)
        		(q_1) edge              node        {$b$} (q_2)
        		edge [bend right] node [swap] {$a$} (q_0)
        		(q_2) edge [loop below] node        {$a,b$} ()
        		;
      		\end{tikzpicture}
	\end{block}
	}
\end{frame}


% Nur falls noch Bedarf besteht.
% \section{Strukturelle Induktion}

\subsection*{}
\begin{frame}
  \frametitle{Motivation}
  Wir wollen die vollständige Induktion "verallgemeinern".
  Vorgehen:
  \begin{enumerate}
    \item Aussage für "grundlegende Elemente" beweisen.
    \item Beweisen das die Aussage auch für daraus zusammen gesetzte Elemente gilt.
  \end{enumerate}
\end{frame}
\begin{frame}
  \frametitle{Beispiel}
  Die Menge $M \subseteq \mathbb{Z}^2$ sei wie folgt definiert:
  \begin{itemize}
    \item $(3, 2) \in M$
    \item Wenn $(m, n) \in M$, dann ist auch $(3m - 2n, m) \in M$
    \item Keine weiteren Elemente liegen in $M$.
  \end{itemize}
  \begin{exampleblock}{Aufgabe}
    Zeigen Sie: $\forall m \in M: \exists k \in \mathbb{N}_0: m = (2^{k+1}+1, 2^k + 1)$
  \end{exampleblock}
\end{frame}


\section{Abschluss}
\subsection{Zusammenfassung}
\begin{frame}
  \frametitle{Was ihr jetzt wissen sollte.}
  \begin{enumerate}
    \item bla
  \end{enumerate}
\end{frame}

\subsection{xkcd}
\begin{frame}[plain]
  \begin{figure}
    \begin{center}
      \includegraphics[width=300pt]{cautionary}
    \end{center}
    {\tiny This really is a true story, and she doesn't know I put it in my comic because her wifi hasn't worked for weeks.}
  \end{figure}
\end{frame}

\end{document}
