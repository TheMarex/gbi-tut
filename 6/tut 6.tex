\documentclass{beamer}
\usepackage[T1]{fontenc}
\usepackage[utf8]{inputenc}
\usepackage{lmodern}
\usepackage[ngerman]{babel}

\usepackage{graphics}

\usepackage{amsmath}
\usepackage{booktabs}
\usepackage{listings}

\usepackage{dot2texi}
\usepackage{tikz}
\usetikzlibrary{arrows,shapes}
 \usetikzlibrary{matrix}
\usetikzlibrary{automata}

\usepackage{wrapfig}

\usetheme{Singapore}
\usecolortheme{dove}
\graphicspath{{images/}{../comics/}}

\newcommand{\hiddencell}[2]{\action<#1->{#2}}

\newcommand{\link}[2][blue]{\underline{\textcolor{#1}{#2}}} 	% \link{www.example.org} zeigt www.example.org in blau und unterstrichen
\renewcommand{\emph}[1]{\textit{\textcolor{blue}{#1}}}			% \emph{Test} zeigt Text in blau und kursiv
\newcommand{\warn}[1]{\textcolor{red}{#1}}			% \warn{Warning} zeigt Warning in rot und kursiv

\AtBeginSection[]
{
  \begin{frame}[plain]
    \frametitle{}
    {\footnotesize
      \tableofcontents[currentsection]
    }
  \end{frame}
}

\title{Grundbegriffe der Informatik}
\author{Patrick Niklaus}

\begin{document}
\begin{frame}
  \frametitle{Grundbegriffe der Informatik}
  \framesubtitle{6. Tutorium}
  \begin{description}
    \item \textbf{Name:} Patrick Niklaus
    \item \textbf{E-Mail:} patrick.niklaus@student.kit.edu
    \item \textbf{Nr:} 43
  \end{description}
\end{frame}

\section{Übungsblatt}
\begin{frame}
  \frametitle{Anmerkungen zum letzten Übungsblatt}
  \begin{enumerate}
    \item Kein "Beweis durch Beispiel". Beweise auf grafischen Darstellungen von Funktionen sind nicht allgemein!
    \item Definitionen aus der Aufgabenstellung müssen nicht bewiesen werden.\\
          (Manchmal muss man allerdings "'wohldefiniertheit"' zeigen.)
  \end{enumerate}
\end{frame}

\section{Graphen im Rechner}
\subsection{Möglichkeiten}
\begin{frame}[fragile]
	\frametitle{Repräsentation von Graphen}
	\begin{block}{Einstieg}
		\begin{small}
		Wie könnte man Graphen im Rechner darstellen?\\
		Zum Beispiel diesen hier:
		\end{small}
		\begin{figure}
			\begin{minipage}{5.5cm}
				\begin{tikzpicture}[auto,node distance=0.5cm, semithick]
					\begin{dot2tex}[styleonly,codeonly, fdp]
						digraph G {
							d2ttikzedgelabels = true;
							node [style="state"];
							edge [];
							0->1
							0->3
							2->1 [topath="bend left"]
							2->2 [topath="loop above"]
							3->2
							2->3
						}
					\end{dot2tex}
				\end{tikzpicture}
			\end{minipage}
		\end{figure}
	\end{block}
\end{frame}

\begin{frame}[fragile]
	\frametitle{Adjazenzliste}
	\begin{wrapfigure}{r}{7cm}
		\vspace{-1cm}
			\begin{tikzpicture}[auto,node distance=0.5cm, semithick]
				\begin{dot2tex}[styleonly,codeonly, fdp]
					digraph G {
						d2ttikzedgelabels = true;
						node [style="state"];
						edge [];
						0->1
						0->3
						2->1 [topath="bend left"]
						2->2 [topath="loop above"]
						3->2
						2->3
					}
				\end{dot2tex}
			\end{tikzpicture}
	\end{wrapfigure}
	Man listet für jeden Knoten dessen benachbarte Knoten auf:\\
	\vspace{.5cm}
	\begin{tabular}{cl}
		\toprule
		$0$&\visible<2->{$\{1, 3\}$}\\
		\midrule
		$1$&\visible<3->{$\{\}$}\\
		\midrule
		$2$&\visible<4->{$\{1,2,3\}$}\\
		\midrule
		$3$&\visible<5->{$\{2\}$}\\
		\bottomrule
	\end{tabular}
\end{frame}

\begin{frame}[fragile]
	\frametitle{Adjazenzmatrix}
	\begin{wrapfigure}{r}{7cm}
		\vspace{-2cm}
		\begin{minipage}{7cm}
			\begin{tikzpicture}[auto,node distance=0.5cm, semithick]
				\begin{dot2tex}[styleonly,codeonly, fdp]
					digraph G {
						d2ttikzedgelabels = true;
						node [style="state"];
						edge [];
						0->1
						0->3
						2->1 [topath="bend left"]
						2->2 [topath="loop above"]
						3->2
						2->3
					}
				\end{dot2tex}
			\end{tikzpicture}
		\end{minipage}
	\end{wrapfigure}
	\begin{tabular}{c||cccc}
		\toprule
		&0&1&2&3\\
		\toprule
		0&\pause0&1&0&1\\
		\midrule
		1&\pause0&0&0&0\\
		\midrule
		2&\pause0&1&1&1\\
		\midrule
		3&\pause0&0&1&0\\
		\bottomrule
	\end{tabular}\\
	\pause
	\vspace{.5cm}
	$A=\begin{pmatrix}0&1&0&1\\0&0&0&0\\0&1&1&1\\0&0&1&0\end{pmatrix}$
\end{frame}
\begin{frame}
	\frametitle{Adjazenzmatrix}
	\begin{definition}
		$A_{ij} = \begin{cases}1&\text{falls } (i,j)\in E\\0&\text{falls } (i,j)\notin E\end{cases}$
	\end{definition}
\end{frame}

\subsection{Anwendung}
\begin{frame}
\frametitle{Liste vs. Matrix}
	\begin{block}{Adjazenzlisten}
	\begin{itemize}
		\visible<2->{\item einfacher Zugriff auf alle adjazenten Knoten}
		\visible<3->{\item um zu überprüfen, ob eine Kante existiert, muss man eventuell alle Nachbarn durchgehen}
	\end{itemize}
	\end{block}

	\begin{block}{Adjanzenzmatrixen}
	\begin{itemize}
		\visible<4->{\item schnelle Überprüfung, ob eine Kante zwischen zwei Knoten $i$ und $j$ existiert}
		\visible<5->{\item um auf einen Nachbarn zuzugreifen, muss man eventuell alle Knoten durchgehen}
	\end{itemize}
	\end{block}
\end{frame}
\begin{frame}
\frametitle{Liste vs. Matrix}
	\begin{block}{Welche Darstellungsform ist geeigneter?}
		Für einen...
		\begin{itemize}
			\visible<1->{\item vollständigen Graphen? \visible<2->{\\\emph{Adjazenzmatrix}}}\pause
			\item Graphen mit nur wenigen Kanten? \visible<3->{\\\emph{Adjazenzliste}}\pause
		\end{itemize}
	\end{block}
\end{frame}


\section{Wegematrix}
\begin{frame}
	\frametitle{Wegematrix}
	\begin{definition}
    So wie Adjazenzmatrizen die Kantenrelation $E$ darstellen wollen wir jetzt die Erreichbarkeitsrelation $E^*$ darstellen.\\
		$E^* := \bigcup \limits^{\infty}_{i=0} E^i $. \\
    Wir nennen die Matrixdarstellung von $E^*$ die \emph{Wegematrix} $W$.
   \begin{displaymath}
     W_{ij} :=
      \begin{cases}
        1, & \text{falls es in G einen Pfad von i nach j gibt} \\
        0, & \text{falls es in G keinen Pfad von i nach j gibt}
      \end{cases}
    \end{displaymath}
	\end{definition}
\end{frame}

\begin{frame}
	\frametitle {Beispiel}
	\begin{columns}
		\column{.5\textwidth}
		\begin{center}
		\begin{tikzpicture}
		  \tikzstyle{every node}=[draw,shape=circle];
			\path[fill] (0,0)  node[circle] (0) {0};
			\path[fill] (2,0)  node[circle] (1) {1};
			\path[fill] (4,0)  node[circle] (2) {2};
			\path[fill] (2,-2) node[circle] (3) {3};

			\path[->,draw] (0) -- (1);
			\path[->,draw] (2) -- (1);
			\path[->,draw] (2) edge [loop right] ();
			\path[->,draw] (3) .. controls (3,-0.5) ..  (2);
			\path[->,draw] (2) .. controls (3,-1.5) ..  (3);
			\path[->,draw] (0) -- (3);
		\end{tikzpicture}
		\end{center}
    \column{.5\textwidth}
        $A = $
				$\left(
						\begin{matrix}
							0 & 1 & 0 & 1 \\
							0 & 0 & 0 & 0 \\
							0 & 1 & 1 & 1 \\
							0 & 0 & 1 & 0
						\end{matrix}
				\right)$
	\end{columns}
	\begin{block}{Fragen}
		\begin{itemize}
			\item Wie sieht die Wegematrix zum oben gezeigten Graph aus?
			\visible<2->{\item Wie sieht die Wegematrix für eine vollständig mit 1en gefüllte Matrix aus?}
			\visible<3->{\item Wann gilt allgemein $W=A$? Wann gilt $E^1=A$?}
		\end{itemize}
	\end{block}
\end{frame}

\subsection{Algorithmen}

\begin{frame}[fragile]
	\frametitle{Algorithmus zur Wegematrix}
  \begin{lstlisting}[language = Java,mathescape,morekeywords={set}]
    // Eingabe: A: Adjazenzmatrix
    // Ausgabe: W Wegematrix

    W = 0 // Nullmatrix
    for i = 0 to n - 1 do
      M = Id // Einheitsmatrix
      for j = 1 to i do
        M = M $\cdot$ A
      od
      W = W + M
    od
    W = sgn$(W)$
  \end{lstlisting}
\end{frame}

\begin{frame}[fragile]
	\frametitle{Algorithmus zur Wegematrix}
  \begin{lstlisting}[language = Java,mathescape,morekeywords={set}]
    // Eingabe: A: Adjazenzmatrix
    // Ausgabe: W Wegematrix

    W = 0 // Nullmatrix
    M = Id // Einheitsmatrix
    for i = 0 to n - 1 do
      W = W + M
      M = M $\cdot$ A
    od
    W = sgn$(W)$
  \end{lstlisting}
\end{frame}

\begin{frame}
	\frametitle{Aufwand}
	\begin{block}{Zählweise}
		Beim Vergleich verschiedener Algorithmen in Bezug auf den Aufwand, sucht man nach einem Maß für die Anzahl der Rechenoperationen für eine Aufgabe der Größe $n$.
	\end{block}
	\begin{block}{Beispiel} \pause
		Summe aller Zahlen von $1$ bis $n$: \\
			$\sum^n_{i=0} i = $\pause$n*(n+1)/2$
	\end{block}
\end{frame}

\begin{frame}
	\begin{block}{Aufwandsvergleich}
		Wenn man in einem bestimmten Zeitraum mit dem $n^5$ Algorithmus gerade noch die Problemgröße $n=1000$ schafft:
	  \begin{itemize}
		  \item Wie große Probleminstanzen schafft man mit dem $n^4$ Algorithmus? \pause
		  \item ... oder mit dem $log_2(n) \cdot n^3$ Algorithmus?
	  \end{itemize}
	\end{block}
\end{frame}

\section{Warshall-Algorithmus}
\subsection*{}
\begin{frame}
	\frametitle{Reflexive Transitive Hülle nach Warshall}
	Wir definieren eine Relation $\sigma^{(k)}$ über $V=\{0,\dots,n-1\}$ mit:
		$\sigma^{(k)} = \{(i,j) \in V \times V |$
    $ \exists p = (i, v_1, \cdots, v_n, j) \in V^+$ mit:\\
    \hfill alle inneren Knoten $v_i \in \{0, \cdots, k\}\}$

	\begin{columns}
	\column{.2\textwidth}
		\begin{tikzpicture}
		  \tikzstyle{every node}=[draw,shape=circle];
			\path[fill] (0,0)  node[circle] (0) {0};
			\path[fill] (1.5,0)  node[circle] (1) {1};
			\path[fill] (1.5,1.5)  node[circle] (2) {2};
			\path[fill] (0,1.5)  node[circle] (3) {3};

			\path[->,draw] (0) edge [loop below] ();
			\path[->,draw] (1) edge [loop below] ();
			\path[->,draw] (2) edge [loop above] ();
			\path[->,draw] (3) edge [loop above] ();
			\path[->,draw] (3) -- (0);
			\path[->,draw] (0) -- (2);
			\path[->,draw] (2) -- (1);
			\visible<2->{\path[->,draw,orange] (3) -- (2);}
			\visible<6->{\path[->,draw,blue] (3) -- (1);}
			\visible<6->{\path[->,draw,blue] (0) -- (1);}
		\end{tikzpicture}
	\column{.4\textwidth}
		\[
		{\color{orange} W_{(0)}}=\
		\alt<2->{
			\left(
			\begin{matrix}
			1 & 0 & 1 & 0\\
			0 & 1 & 0 & 0\\
			0 & 1 & 1 & 0\\
			1 & 0 & {\color{orange}1} & 1
			\end{matrix}
			\right)
		}{
			\left(
			\begin{matrix}
			1 & 0 & 1 & 0\\
			0 & 1 & 0 & 0\\
			0 & 1 & 1 & 0\\
			1 & 0 & 0 & 1
			\end{matrix}
			\right)
		}
		\]
		\[
		\visible<5->{
			{ \color{blue} W_{(2)}}=\
		}
		\alt<-5>{
			\visible<5->{
				\left(
				\begin{matrix}
				1 & 0 & 1 & 0\\
				0 & 1 & 0 & 0\\
				0 & 1 & 1 & 0\\
				1 & 0 & {\color{orange}1} & 1
				\end{matrix}
				\right)
			}
		}
		{
			\left(
			\begin{matrix}
			1 & {\color{blue} 1} & 1 & 0\\
			0 & 1 & 0 & 0\\
			0 & 1 & 1 & 0\\
			1 & {\color{blue} 1} & {\color{orange}1} & 1
			\end{matrix}
			\right)
		}
		\]
	\column{.4\textwidth}
		\[
		\visible<3->{
			{ \color{green} W_{(1)}}=\
		}
		\alt<-3>{
			\visible<3->{
				\left(
				\begin{matrix}
				1 & 0 & 1 & 0\\
				0 & 1 & 0 & 0\\
				0 & 1 & 1 & 0\\
				1 & 0 & {\color{orange}1} & 1
				\end{matrix}
				\right)
			}
		}
		{
			\left(
			\begin{matrix}
			1 & 0 & 1 & 0\\
			0 & 1 & 0 & 0\\
			0 & 1 & 1 & 0\\
			1 & 0 & {\color{orange}1} & 1
			\end{matrix}
			\right)
		}
		\]
		\[
		\visible<7->{
			{ \color{red} W_{(3)}}=\
		}
		\alt<-7>{
			\visible<7->{
				\left(
				\begin{matrix}
				1 & {\color{blue} 1} & 1 & 0\\
				0 & 1 & 0 & 0\\
				0 & 1 & 1 & 0\\
				1 & {\color{blue} 1} & {\color{orange}1} & 1
				\end{matrix}
				\right)
			}
		}
		{
			\left(
			\begin{matrix}
			1 & {\color{blue} 1} & 1 & 0\\
			0 & 1 & 0 & 0\\
			0 & 1 & 1 & 0\\
			1 & {\color{blue} 1} & {\color{orange}1} & 1
			\end{matrix}
			\right)
		}
		\]
	\end{columns}
\end{frame}

\begin{frame}[fragile]
	\frametitle{Der Warshall-Algorithmus}
  \begin{lstlisting}[language = Java,mathescape,morekeywords={set}]
    \\ Eingabe: A Adjanzenzmatrix
    \\ Ausgabe: W Wegematrix
    $W$ := $A$
    for i=0 to n-1 do
      $W[i,i]$ = $1$

    for k=0 to n-1 do
      for i=0 to n-1 do
        for j=0 to n-1 do
          $W[i,j] = max\{W[i,j], min\{W[i,k], W[k, j]\}\}$
  \end{lstlisting}
\end{frame}

\subsection{Aufgaben}
\begin{frame}[fragile]
  \frametitle{Aufgaben}
      Berechnet mit dem Algorithmus von Warshall die Wegematrix.\\
      Gebt nach jedem Durchlauf der äußersten Schleife $W_k$ an.
      \begin{center}
        \begin{tikzpicture}[->,>=stealth,baseline=-5mm]
          \matrix[matrix of math nodes,nodes={draw,circle,minimum size=5mm,inner sep=2pt},row sep=10mm,column sep=10mm]
          {
            |(0)| 0 & |(1)| 1 & |(2)| 2 \\
            & |(3)| 3 & \\
          };
          \draw  (0) -- (1);
          \draw  (2) -- (1);
          \draw  (1) to [bend left] (3);
          \draw  (3) to [bend right] (2);
          \path (1) edge [loop above] ();
        \end{tikzpicture}
      \end{center}
\end{frame}

\begin{frame}
	\frametitle{Klausuraufgabe WS 2010/2011 - Nr.2}
	Für $n\in \mathbb{N}_0, n \geq 2$ sei ein Graph $U_n=(\mathbb{G}_{2n},E_n)$ definiert mit Kantenmenge $E_n=\{\{x,y\}|\text{ggT}(x+y,2n)=1\}$.\\
	\vspace{.2cm}
	\begin{exampleblock}{Zur Erinnerung:}
		\begin{itemize}
			\item $\forall m\in \mathbb{N}_0: \mathbb{G}_m=\{i|0\leq i<m\}$
			\item $\text{ggT}(x,y)$ ist der kleinste gemeinsame Teiler von $x$ und $y$
		\end{itemize}
	\end{exampleblock}
	\begin{description}
		\item[a)] Zeichnen Sie die Graphen $U_3$, $U_4$ und $U_5$.
		\item[b)] Geben sie für $U_4$ und $U_5$ jeweils einen Weg an, bei dem der Anfangsknoten gleich dem Endknoten ist und jeder andere Knoten des Graphen genau einmal in dem Weg vorkommt.
		\item[c)] Geben Sie die Adjazenzmatrix für $U_4$ an.
	\end{description}
\end{frame}


\section{Abschluss}
\subsection{Zusammenfassung}
\begin{frame}
  \frametitle{Was ihr jetzt wissen sollte.}
  \begin{enumerate}
    \item Was ist eine Adjazenzmatrix/Adjazenzliste?
    \item Was sind jeweils die Vor-/Nachteile.
    \item Was ist eine Wegematrix?
    \item Wie sieht ein einfacher Algorithmus aus der die Wegematrix berechnet?
    \item Wie funktioniert der Algorithmus von Warshall?
  \end{enumerate}
\end{frame}

\subsection{xkcd}
\begin{frame}[plain]
  \begin{figure}
    \begin{center}
      \includegraphics[width=200pt]{perspective}
    \end{center}
    {\tiny I wonder what I was dreaming to prompt that. I hope it wasn't the Richard Stallman Cirque de Soleil thing again.}
  \end{figure}
\end{frame}

\end{document}
