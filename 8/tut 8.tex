\documentclass{beamer}
\usepackage[T1]{fontenc}
\usepackage[utf8]{inputenc}
\usepackage{lmodern}
\usepackage[ngerman]{babel}

\usepackage{graphics}

\usepackage{amsmath}
\usepackage{booktabs}
\usepackage{listings}

\usepackage{dot2texi}
\usepackage{tikz}
\usetikzlibrary{arrows,shapes}
 \usetikzlibrary{matrix}
\usetikzlibrary{automata}

\usepackage{wrapfig}

\usetheme{Singapore}
\usecolortheme{dove}
\graphicspath{{images/}{../comics/}}

\newcommand{\xn}{\visible<2->{$\times$}}
\newcommand{\xj}{\visible<2->{$\checkmark$}}
\newcommand{\rd}[1]{\textcolor{red}{#1}}
\newcommand{\gn}[1]{\textcolor{green}{#1}}
\newcommand{\bl}[1]{\textcolor{blue}{#1}}

\newcommand{\hiddencell}[2]{\action<#1->{#2}}

\newcommand{\link}[2][blue]{\underline{\textcolor{#1}{#2}}} 	% \link{www.example.org} zeigt www.example.org in blau und unterstrichen
\renewcommand{\emph}[1]{\textit{\textcolor{blue}{#1}}}			% \emph{Test} zeigt Text in blau und kursiv
\newcommand{\warn}[1]{\textcolor{red}{#1}}			% \warn{Warning} zeigt Warning in rot und kursiv

\AtBeginSection[]
{
  \begin{frame}[plain]
    \frametitle{}
    {\footnotesize
      \tableofcontents[currentsection]
    }
  \end{frame}
}

\title{Grundbegriffe der Informatik}
\author{Patrick Niklaus}

\begin{document}
\begin{frame}
  \frametitle{Grundbegriffe der Informatik}
  \framesubtitle{8. Tutorium}
  \begin{description}
    \item \textbf{Name:} Patrick Niklaus
    \item \textbf{E-Mail:} patrick.niklaus@student.kit.edu
    \item \textbf{Nr:} 43
  \end{description}
\end{frame}

%\section{Übungsblatt}
%\begin{frame}
%  \frametitle{Anmerkungen zum letzten Übungsblatt}
%  \begin{enumerate}
%    \item Bla
%  \end{enumerate}
%\end{frame}

\section{Automaten}
\subsection{Mealy-Automaten}
\begin{frame}
	\frametitle{Mealy-Automaten}
	\begin{definition}
		Ein (endlicher) Mealy-Automat $A=(Z, z_0, X, f, Y, g)$ ist festgelegt durch:
		\begin{description}
			\item[$Z$] endliche Zustandsmenge
			\item[$z_0\in Z$] Startzustand
			\item[$X$] Eingabealphabet
			\pause
			\item[$f:Z\times X\to Z$] Zustandsüberführungsfunktion
			\pause
			\item[$Y$] Ausgabealphabet
			\pause
			\item[$g:Z\times X\to Y^*$] Ausgabefunktion
		\end{description}
	\end{definition}
	
	\begin{tikzpicture}[->,>=stealth',shorten >=1pt,auto,node distance=2.8cm,semithick]
	\tikzstyle{every state}=[fill=white,draw=black,text=black]

	\node[initial,state, initial text= ] (A)                    {$z_0$};
	\node[state]         (B) [right of=A] {$z_1$};

	\path	(A) edge [loop above] node {b|1} (A)
				edge [bend left]  node {a|0} (B)
			(B) edge [loop above] node {b|1} (B)
				edge [bend left]  node {a|2} (A);
	\end{tikzpicture}
\end{frame}
\begin{frame}
	\frametitle{Von $f$ zu $f^*$ zu $f^{**}$}
	\only<1-4>{
		\begin{definition}
			$f(z,x)=z'$\\
			\pause
			\textbf{In Worten:} $z'$ ist der Zustand, der dem Tupel $(z,x)$ durch die Zustandsüberführungsfunktion $f$ zugeordnet wird.
		\end{definition}
		\pause
	}
	\begin{definition}
		\vspace{-.5cm}
		\begin{eqnarray*}
			f^*(z,\varepsilon)&=&z\\
			\forall w\in X^*:\forall x\in X: f^*(z,wx)&=&f(f^*(z,w),x)
		\end{eqnarray*}
		\pause
		\textbf{In Worten:} $f^*$ ordnet dem Tupel $(z,w)$ einen Zustand zu, in dem der Automat sich nach Eingabe des Wortes $w$ befindet.
	\end{definition}
	\pause
	\only<5->{
		\visible<6->{
			\begin{definition}
				\vspace{-.5cm}
				\visible<6->{
					\begin{eqnarray*}
						f^{**}(z,\varepsilon)&=&z\\
						\forall w\in X^*:\forall x\in X: f^{**}(z,wx)&=&f^{**}(z,w)\cdot f(f^*(z,w),x)
					\end{eqnarray*}
				}
				\visible<7->{
					\textbf{In Worten:} $f^{**}$ ordnet dem Tupel $(z,w)$ die Konkatenation der Zustände zu, die der Automat bei Eingabe des Wortes $w$ durchläuft.
				}
			\end{definition}
		}
	}
\end{frame}
\begin{frame}
	\begin{center}
		\includegraphics{getraenkeautomat}
	\end{center}

	\begin{block}{Aufgabe}
		Was sind $f((0,-),Z),f^*((0,-),R1O),f^{**}((0,-),R1O)$ im Getränkeautomat im Graphen und rechnerisch?
	\end{block}
\end{frame}
\begin{frame}
	\begin{exampleblock}{Lösung}
		\begin{itemize}
			\item $f((0,-),Z)$ überführt den Zustand $(0,-)$ in den Zustand $(0,Z)$.
			\item $f((0,-),Z)=(0,Z)$
		\end{itemize}
	\end{exampleblock}
\end{frame}
\begin{frame}
	\begin{exampleblock}{Lösung}
		\begin{itemize}
			\item $f^*((0,-),R1O)$ ist der Zustand, zu dem man gelangt, indem man nacheinander $R, 1$ und $O$ eingibt.
		\end{itemize}
		\begin{eqnarray*}
			f^*((0,-),R1O)&=&f(f^*((0,-),R1),O)\\
						&=&f(f(f^*((0,-),R),1),O)\\
						&=&f(f(f(f^*((0,-),\varepsilon),R),1),O)\\
						&=&f(f(f((0,-),R),1),O)\\
						&=&f(f((0,R),1),O)\\
						&=&f((1,R),O)\\
						&=&(0,-)
		\end{eqnarray*}

	\end{exampleblock}
\end{frame}
\begin{frame}
	\begin{exampleblock}{Lösung}
		\begin{itemize}
			\item $f^{**}((0,-),R1O)$ ist die Konkatenation der Zustände, die man nach Eingabe von $R, 1$ und $O$ erhält.
		\end{itemize}
		\begin{eqnarray*}
			f^{**}((0,-),R1O)&=&f^{**}((0,-),R1)\cdot f(f^*((0,-),R1),O)\\
						&\overset{s.o.}{=}&f^{**}((0,-),R1)\cdot (0,-)\\
						&=&f^{**}((0,-),R)\cdot f(f^*((0,-),R),1)\cdot (0,-)\\
						&\overset{s.o.}{=}&f^{**}((0,-),R)\cdot (1,R)\cdot (0,-)\\
						&=&f^{**}((0,-),\varepsilon)\cdot f(f^*((0,-),\varepsilon),R)\cdot (1,R)\cdot (0,-)\\
						&=&(0,-)\cdot f((0,-),R)\cdot (1,R)\cdot (0,-)\\
						&=&(0,-)\cdot (0,R)\cdot (1,R)\cdot (0,-)
		\end{eqnarray*}

	\end{exampleblock}
\end{frame}

\begin{frame}
	\begin{center}
		\includegraphics{getraenkeautomat2}
	\end{center}

	\begin{block}{Aufgabe}
		Was sind $g((0,-),Z),g^*((0,-),11ZO),g^{**}((0,-),11ZO)$ im Getränkeautomat im Graphen und rechnerisch?
	\end{block}
\end{frame}

\begin{frame}
	\begin{exampleblock}{Lösung}
		\begin{itemize}
			\item $g((0,-),Z)$ gibt $\varepsilon$ aus, da das an der Kante $((0,-),(0,Z))$ definiert ist.
			\item $g((0,-),Z)=\varepsilon$
		\end{itemize}
	\end{exampleblock}
\end{frame}
\begin{frame}
	\begin{exampleblock}{Lösung}
		\begin{itemize}
			\item $g^*((0,-),11ZO)$ gibt $1$ aus, da eine Münze eingeworfen wurde und durch "`Cancel"' wieder ausgegeben wird.
			\item $g^*((0,-),11ZO)=\cdots$ Vorgehen analog zu $f^*\cdots = 1$
		\end{itemize}
	\end{exampleblock}
	\pause
	\begin{exampleblock}{Lösung}
		\begin{itemize}
			\item $g^{**}((0,-),11ZO)$ gibt $1$ aus, da eine Münze eingeworfen wurde und durch "`Cancel"' wieder ausgegeben wird.
			\item $g^{**}((0,-),11ZO)=\cdots$ Vorgehen analog zu $f^{**}\cdots = 1Z$
		\end{itemize}
	\end{exampleblock}
\end{frame}

\section{Moore-Automaten}
\begin{frame}[plain]
	\frametitle{Moore-Automaten}
	\begin{definition}
		Ein (endlicher) Mealy-Automat $A=(Z, z_0, X, f, Y, g)$ ist festgelegt durch:
		\begin{description}
			\item[$Z$] endliche Zustandsmenge
			\item[$z_0\in Z$] Startzustand
			\item[$X$] Eingabealphabet
			\pause
			\item[$f:Z\times X\to Z$] Zustandsüberführungsfunktion
			\pause
			\item[$Y$] Ausgabealphabet
		\end{description}
	\end{definition}
	\begin{alertblock}{Unterschied zu Mealy-Automaten}
		Ausgabe hängt nur vom Zustand ab (Mealy: von Zustand und Eingabe).
	\end{alertblock}

	\begin{tikzpicture}[->,>=stealth',shorten >=1pt,auto,node distance=2.8cm,semithick]
	\tikzstyle{every state}=[fill=white,draw=black,text=black]

	\node[initial,state, initial text= ] (A)    {$z_0|1$};
	\node[state]         (B) [right of=A] {$z_1|0$};

	\path	(A) edge [bend left]  node {a,b} (B)
			(B) edge [loop right] node {b} (B)
				edge [bend left]  node {a} (A);
	\end{tikzpicture}
\end{frame}

\begin{frame}[fragile]
	\frametitle{Aufgabe zum Schluss}
 	\begin{tikzpicture}[->,>=stealth',shorten >=1pt,auto,node distance=2.8cm,semithick]
	\tikzstyle{every state}=[fill=white,draw=black,text=black]

		\node[initial,state, initial text= ] (A) {$q_{\varepsilon}|0$};
		\node[state]         (B) [above right of=A] {$q_a|0$};
		\node[state]         (C) [below right of=A] {$q_b|0$};
		\node[state]         (D) [below right of=B] {$q_f|1$};
		\node[state]         (E) [right of=D] {$q_r|0$};

		\path	(A) edge node {a} (B)
					edge node {b} (C)
				(B) edge [loop above] node {a} (B)
					edge node {b} (D)
				(C) edge [loop below] node {b} (C)
					edge node {a} (D)
				(D) edge node {a,b} (E)
				(E) edge [loop right] node {a,b} (E);
	\end{tikzpicture}
	\begin{exampleblock}{Aufgabe}
		Was gibt obiger Automat für ein beliebiges Wort aus $L_1=\{a,b\}^*$ aus?
	\end{exampleblock}

\end{frame}


\section{Abschluss}
\subsection{Zusammenfassung}
\begin{frame}
  \frametitle{Was ihr jetzt wissen sollte.}
  \begin{enumerate}
    \item Was ist ein Mealy-Automat?
    \item Was ist ein Moore-Automat?
    \item Wie repräsentiert man diese Automaten grafisch?
  \end{enumerate}
\end{frame}

\subsection{xkcd}
\begin{frame}[plain]
  \begin{figure}
    \begin{center}
      \includegraphics[width=300pt]{cautionary}
    \end{center}
    {\tiny This really is a true story, and she doesn't know I put it in my comic because her wifi hasn't worked for weeks.}
  \end{figure}
\end{frame}

\end{document}
