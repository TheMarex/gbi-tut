\documentclass{beamer}
\usepackage[T1]{fontenc}
\usepackage[utf8]{inputenc}
\usepackage{lmodern}
\usepackage[ngerman]{babel}

\usepackage{graphics}

\usepackage{amsmath}
\usepackage{booktabs}
\usepackage{listings}

\usepackage{dot2texi}
\usepackage{tikz}
\usetikzlibrary{arrows,shapes}
 \usetikzlibrary{matrix}
\usetikzlibrary{automata}

\usepackage{wrapfig}

\usetheme{Singapore}
\usecolortheme{dove}
\graphicspath{{images/}{../comics/}}

\newcommand{\xn}{\visible<2->{$\times$}}
\newcommand{\xj}{\visible<2->{$\checkmark$}}
\newcommand{\rd}[1]{\textcolor{red}{#1}}
\newcommand{\gn}[1]{\textcolor{green}{#1}}
\newcommand{\bl}[1]{\textcolor{blue}{#1}}

\newcommand{\hiddencell}[2]{\action<#1->{#2}}

\newcommand{\link}[2][blue]{\underline{\textcolor{#1}{#2}}}
\renewcommand{\emph}[1]{\textit{\textcolor{gray}{#1}}}
\newcommand{\warn}[1]{\textcolor{red}{#1}}
\newcommand{\ans}[2]{\visible<#1->{\textcolor{green!70!black}{#2}}}
\newcommand{\wrong}[2]{\visible<#1->{\textcolor{red!70!black}{#2}}}
\newcommand{\xb}[1]{{\bf #1}}   % \xb {fett}
\newcommand{\xu}{\underline}    % \xu {unterstrichen}
\newcommand{\hide}{\onslide+<+(1)->}

\AtBeginSection[]
{
  \begin{frame}[plain]
    \frametitle{}
    {\footnotesize
      \tableofcontents[currentsection]
    }
  \end{frame}
}

\title{Grundbegriffe der Informatik}
\author{Patrick Niklaus}

\begin{document}
\begin{frame}
  \frametitle{Grundbegriffe der Informatik}
  \framesubtitle{11. Tutorium}
  \begin{description}
    \item \textbf{Name:} Patrick Niklaus
    \item \textbf{E-Mail:} patrick.niklaus@student.kit.edu
    \item \textbf{Nr:} 43
  \end{description}
\end{frame}

\section{Übungsblatt}
\begin{frame}
  \frametitle{Anmerkungen zum letzten Übungsblatt}
  \begin{enumerate}
    \item Achtet auf die Beschriftung euerer Graphen!
    \item Immer Sonderfälle beachten!
    \item Bei formalen Beweisen, immer auf die \emph{genaue} Definition der Objekte achten.
  \end{enumerate}
\end{frame}

\section{Entscheidbarkeit}
\subsection{Motivation}
\begin{frame}
  \frametitle{Motivation}
  Wir haben Turingmachinenakzeptoren kennengelernt, die eine Sprache $L(T)$ akzeptieren.
  \begin{alertblock}{}
    Gibt es so einen TMA zu \emph{jeder} Sprache?
    \begin{center} \warn{Nein.} \end{center}
  \end{alertblock}
  Und das kann man einfach einsehen.
\end{frame}
\subsection{Kodierte Turingmachinen}
\begin{frame}
  \frametitle{Beschreibung einer Turingmachine}
  Wir kennen zwei Arten von Beschreibungen: Graphen und Tabellen
  \begin{center}
    \begin{tabular}{cccccc}
      \toprule
      & r & $c_0$ & $c_1$ & h \\
      \midrule
      0 & r,0,R   & $c_0$,0,L & $c_0$,1,L \\
      1 & r,1,R   & $c_0$,1,L & $c_1$,0,L \\
      $\square$  & $c_1$,$\square$, L & h,$\square$ ,R   & $c_0$,1,L & \hphantom{C,1,L} \\
      \bottomrule
    \end{tabular}
  \end{center}
  Jede beschreibt eine TM \emph{eindeutig}. Unser Ziel: Die Tabelle als Folge von Symbolen kodieren.
\end{frame}
\begin{frame}
  \frametitle{Kodierung einer Turingmachine}
  Jeder Eintrag in der Tabelle ist beschreibar durch $(z, x, z', y, m)$
  \begin{description}
    \item[$z \in Z$:] Aktueller Zustand
    \item[$x \in X$:] Gelesenes Symbol auf dem Band
    \item[$z' \in Z$:] Nächster Zustand
    \item[$y \in X$:] Ausgabesymbol auf dem Band
    \item[$m \in M$:] $M = \{L, 0, R\}$ Kopfbewegung
  \end{description}
  \begin{block}{}
    $Z$, $X$, $M$ sind alle \emph{endlich} also: Durchnummerieren.
  \end{block}
\end{frame}
\begin{frame}
  \frametitle{Kodierte Darstellung}
  Jede Vorschrift der TM ist jetzt als Zahlenfolge darstellbar.
  \begin{block}{Was fehlt:}
    Ene Zahlenkodierung (dezimal, binär, unär, etc.) und ein Format.
  \end{block}
  \begin{alertblock}{}
  Bei einer unären Kodierung mit Einsblöcken als Trennsymbole ist die \emph{kodierte} TM selbst wieder eine Binärzahl!
  \end{alertblock}
\end{frame}
\begin{frame}
  \frametitle{Gödelnummer}
  Man kann jeder TM eindeutig als natürliche Zahl kodieren. Dies nennt man \emph{Gödelnummer} einer TM.
  Es gibt also eine Funktion $f(T) \mapsto n_T \in \mathbb{N}$.
  \begin{alertblock}{}
    $\mathbb{N}$ ist abzählbar unendlich. Es gibt also höhsten \emph{abzählbar unendlich} viele Turingmachinen.
  \end{alertblock}
\end{frame}
\subsection{Schneller Beweis}
\begin{frame}
  \frametitle{Existenz unendscheidbarer Sprachen}
  \begin{theorem}
    Es gibt mehr formale Sprachen als Turingmachinenakzeptoren.
  \end{theorem}
  \begin{proof}
    Sei $A$ ein endliches Alphabet. $A^*$ ist die Menge aller Wörter,
    also ist $\mathcal{P}(A^*)$ die Menge aller Sprachen. $A^*$ ist abzählbar unendlich.
    \begin{block}{Nach Cantor gilt:}
    Die \emph{Potenzmenge} einer abzählbaren Menge ist überabzählbar.\\
    $\Rightarrow$ überabzählbar viele Sprachen.\\
    \xb{Aber:} Es gibt nur abzählbar viele Turingmachinenakzeptoren.
    \end{block}
  \end{proof}
  Da jeder TMA nur genau eine Sprache erkennt, gibt es also Sprachen die von einem TMA nicht erkannt werden können.
\end{frame}
\subsection{Halteproblem}
\begin{frame}
  \frametitle{Halteproblem}
  Wir wollen wissen: Kann man \emph{allgemein} mit einer Turingmachine feststellen, ob eine andere Turingmachien hält?
  \begin{definition}
    $H := \{w | w \text{ ist die Kodierung einer TM } T_w \text{ und } T_w(w) \text{ hält }\}$ ist (hier) die Sprache des Halteproblems.
  \end{definition}
  \pause
  \begin{block}{}
  Gibt es einen TMA der entscheidet ob $w \in H$?\\
  Anders: Ist H eine \emph{entscheidbare} Sprache? \warn{\xb{Nein.}}
  \end{block}
  \begin{alertblock}{Wichtig:}
    Uns interessiert nicht ob man für \emph{spezielle} TM berechnen kann ob sie halten oder nicht.
  \end{alertblock}
\end{frame}

\section{Nerode-Äquivalenzrelastion}
\subsection{Äquivalenzrelationen}
\begin{frame}
  \frametitle{Relation als Graph}
    Darstellung von Relationen als gerichtete Graphen $(xRy) \Rightarrow (x, y) \in E$.
    \begin{exampleblock}{Worang erkennt man folgende Eigenschaften einer Relation:}
      \begin{itemize}
        \item Reflexivität?
        \item Symmetrie?
        \item Transitivität?
      \end{itemize}
    \end{exampleblock}
\end{frame}

\subsection{Äquivalenzrelation über Sprachen}
\begin{frame}
  \frametitle{Äquivalenzrelationen von Nerode}
  \begin{definition}
  	für alle $w_1,w_2 \in A^*$ ist\\
  	$w_1 \equiv_L w_2 \Leftrightarrow (\forall w \in A^*: w_1w \in L \Leftrightarrow w_2w \in L)$
  \end{definition}

  \begin{exampleblock}{Erst mal ein Beispiel}
    Sei $L = \langle a*b* \rangle \subset A^*$ Gilt $w_1 \equiv_L w_2$?
      \begin{itemize}
        \item $w_1 = aaa, w_2 = a$ \ans{2}{Richtig}
        \item $w_1 = aaab, w_2 = abb$ \ans{3}{Richtig}
        \item $w_1 = aa, w_2 = abb$ \wrong{4}{Falsch}
        \item $w_1 = aba, w_2 = babb$ \ans{5}{Richtig}
        \item $w_1 = ab, w_2 = ba$ \ans{6}{Falsch}
      \end{itemize}
   \end{exampleblock}
\end{frame}

\subsection{Faktormengen}
\begin{frame}
  \frametitle{Faktormenge}
  \begin{definition}
    Sei $\equiv$ eine Äquivalenzrelationen über einer Menge $M$. Die \emph{Menge alle Äquivalenzklassen} von $\equiv$ über M heißt \emph{Faktormenge}.
  \end{definition}
  \begin{exampleblock}{Etwas Modulo-Arithmetik}
    Division von Ganzzahlen mit Rest durch 5:
    \begin{description}
      \item[Äquivalenzklasse:] $a \in [b] \Leftrightarrow (a \mod 5) = (b \mod 5)$
      \item[Faktormenge:] $\mathbb{Z}/_{5\mathbb{Z}} = \{[0], [1], [2], [3], [4]\}$
    \end{description}
  \end{exampleblock}
\end{frame}
\begin{frame}
  \frametitle{Aufgaben}
    Welche Äquivalenzklassen/Faktormengen haben folgende Äquivalenzrelationen über $\mathbb{Z}$:
    \begin{enumerate}
      \item $a R b \Leftrightarrow \exists k \in \mathbb{Z}: a - b = 3k$\\
        \ans{2}{$F = \{[6], [4], [8]\}$}
      \item $a R b \Leftrightarrow \exists k \in \mathbb{Z}: a - b = 4k$\\
        \ans{3}{$F = \{[0], [1], [2], [3]\}$}
    \end{enumerate}
\end{frame}
\begin{frame}
  \frametitle{Motivation}
  Wir wollen Sprachen \emph{aufteilen} in ``Gruppen'' von Wörtern die ``ähnlich'' sind.

  Die Anzahl dieser ``Gruppen'' wird uns später die \emph{minimale} Anzahl an Zustände verraten, die ein Automat brauch um sie zu erkennen.
\end{frame}
\begin{frame}
  \frametitle{Faktormengen von Sprachen}
  Nutze die Äquivalenzrelastion von Nerode $\equiv_L$ um eine Sprache $L$ in Äquivalenzklassen aufzuteilen.
  \begin{exampleblock}{Beispiel}
    $L = \langle {a*}{(ba)*} \rangle$
    \begin{itemize}
      \item $[a] = \langle {a*} \rangle$
      \item $[ab] = \langle {a*}{(ba)*}b \rangle$
      \item $[aba] = \langle {a*}ba{(ba)*}\rangle$
      \item $[bb] = \langle {(a|b)*}bb{(a|b)*} \; | \; {(a|b)*}{b(a|b)*}{ab(a|b)*} \; | \; {(a|b)*}{b(a|b)*}aa{(a|b)*} \rangle$
    \end{itemize}
  \end{exampleblock}
\end{frame}
\begin{frame}
  \frametitle{Aufgaben}
  Bildet zu den folgenden Sprachen die Nerode-Äquivalenzklassen:
  \begin{enumerate}
    \item $L = \{01, 10\}^*$
    \item $L = \{1^n0^n | n \in \mathbb{N}\}$
    \item $L = \langle aa*bb \rangle$
  \end{enumerate}
\end{frame}


\section{Abschluss}
\subsection{Zusammenfassung}
\begin{frame}
  \frametitle{Was ihr jetzt wissen sollte.}
  \begin{enumerate}
    \item Was ist Entscheidbarkeit?
    \item Ist jede Sprache entscheidbar?
    \item Welche Sprache ist nicht entscheidbar?
    \item Was sind Faktormengen?
    \item Was ist die Nerode-Äquivalenzrelation?
  \end{enumerate}
\end{frame}

\subsection{xkcd}
\begin{frame}[plain]
  \begin{figure}
    \begin{center}
      \includegraphics[width=300pt]{3am.png}
    \end{center}
  \end{figure}
\end{frame}

\end{document}
