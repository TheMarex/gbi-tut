\section{Wörter}
\subsection{Alphabete}
\begin{frame}
  \frametitle{Alphabete}
  \begin{definition}
    Sei A eine endliche Menge von Symbolen. Dann heißt A ein \emph{Alphabet}.
  \end{definition}
  \begin{exampleblock}{Beispiele}
    \begin{itemize}
      \item $A := \{a, b, c\}$
      \item $A := \{\phi, 1, 2\}$
      \item $A := \{l, r, t, o\}$
    \end{itemize}
  \end{exampleblock}
\end{frame}
\subsection{Wörter}
\begin{frame}
  \frametitle{Wörter}
  \begin{definition}
    Sei A ein Alphabet und $w: \mathbb{G}_n \rightarrow A$ eine \emph{surjektive} Abbildung. Dann heißt w ein \emph{Wort über dem Alphabet A}.
  \end{definition}\pause
  \begin{alertblock}{Etwas weniger formal}
    Ein Wort ist eine Folge von Symbolen aus einem Alphabet A.
  \end{alertblock}\pause
  \begin{definition}
    Das Wort $\varepsilon: \{\} \rightarrow \{\}$ heißt das \emph{leere Wort}. \\
    {\tiny Neutrales Element gegenüber der Wortkonkatenation.}
  \end{definition}\pause
  \begin{alertblock}{Erinnerung}
    $\mathbb{G}_n :=  \{k \in \mathbb{N}_0 | 0 \leq k < n\} \Rightarrow \mathbb{G}_0 = \{\}$
  \end{alertblock}
\end{frame}
\begin{frame}
  \frametitle{Wörter}
  \begin{exampleblock}{Beispiele}
    \begin{itemize}
      \item
      $
        w_1: \mathbb{G}_5 \rightarrow \{l, r, t, o\}: w_1(i) \mapsto \left\{
        \begin{array}{l l}
          t, & \quad i = 0\\
          r, & \quad i = 1\\
          l, & \quad i \in \{2, 4\}\\
          o, & \quad i = 3\\
        \end{array} \right.
      $ \\
      wir schreiben auch $w_1 = trlol$
      \item
      $
        w_2: \mathbb{G}_6 \rightarrow \{1, 0\}: w_2(i) \mapsto \left\{
        \begin{array}{l l}
          1, & \quad \text{$i$ gerade}\\
          0, & \quad \text{$i$ ungerade}\\
        \end{array} \right.
      $ \\
      wir schreiben auch $w_2 = 101010$
    \end{itemize}
  \end{exampleblock}
\end{frame}
\subsection{Konkatenation}
\begin{frame}
  \frametitle{Konkatenation von Wörtern}
  \begin{definition}
    Seien $w_1: \mathbb{G}_{n_1} \rightarrow A_1, w_2: \mathbb{G}_{n_2} \rightarrow A_2$ Wörter.\\
    Dann ist $w_1 \cdot w_2: \mathbb{G}_{n_1 + n_2} \rightarrow A_1 \cup A_2,$
    $(w_1 \cdot w_2)(i) \mapsto \left\{
        \begin{array}{l l}
          w_1(i), & \quad 0 \leq i < n_1\\
          w_2(i-n_1), & \quad n_1 \leq i < n_2\\
        \end{array} \right.
    $\\
    die Konkatenation von $w_1$ und $w_2$.
  \end{definition}\pause
  \begin{alertblock}{Etwas weniger formal}
    Wir hängen die Buchstabenfolge $w_1$ an $w_2$ an.
  \end{alertblock}
\end{frame}
\begin{frame}
  \frametitle{Konkatenation von Wörtern}
  \begin{exampleblock}{Beispiele}
    \begin{enumerate}
      \item $w_1 := ab, w_2 := ba$\\
             $w_1 \cdot w_2 = ab \cdot ba = abba$
      \item $w_1 := Hallo, w_2 := We, w_3 := lt$\\
             $w_1 \cdot w_2  \cdot w_3 = Hallo \cdot We \cdot lt = HalloWelt$
      \item $w_1 := 01$\\
             $w_1^3 = w_1 \cdot w_1^2 = w_1 \cdot w_1 \cdot w_1 = 10 \cdot 10 \cdot 10 = 101010$
    \end{enumerate}
  \end{exampleblock}
\end{frame}

\begin{frame}
  \frametitle{Menge aller Wörter}
  \begin{definition}
    Die Menge der Wörter der Länge \emph{n} wird bezeichnet mit $A^n$. Die Menge aller Wörter $A^*$ ist definiert als $A^* = \bigcup \limits^{\infty}_{i=0} A^i$ \emph{(Kleenesche Hülle von A)}.
  \end{definition}
  \begin{exampleblock}{Beispiele}
    \begin{enumerate}
      \item $A := \{a, b\}, A^3 = \{aaa, aab, aba, abb, baa, bab, bba, bbb\}$
      \item $A := \{0, 1\}, A^* = \{\varepsilon, 0, 1, 00, 11, 01, 10, 000, 001, ...\}$
      \item $A := \{a, b, c\}, A^* = \{\varepsilon, a, b, c, aa, ab, ac, bb, ba, ...\}$
    \end{enumerate}
  \end{exampleblock}
\end{frame}

