\section{Graphen im Rechner}
\subsection{Möglichkeiten}
\begin{frame}[fragile]
	\frametitle{Repräsentation von Graphen}
	\begin{block}{Einstieg}
		\begin{small}
		Wie könnte man Graphen im Rechner darstellen?\\
		Zum Beispiel diesen hier:
		\end{small}
		\begin{figure}
			\begin{minipage}{5.5cm}
				\begin{tikzpicture}[auto,node distance=0.5cm, semithick]
					\begin{dot2tex}[styleonly,codeonly, fdp]
						digraph G {
							d2ttikzedgelabels = true;
							node [style="state"];
							edge [];
							0->1
							0->3
							2->1 [topath="bend left"]
							2->2 [topath="loop above"]
							3->2
							2->3
						}
					\end{dot2tex}
				\end{tikzpicture}
			\end{minipage}
		\end{figure}
	\end{block}
\end{frame}

\begin{frame}[fragile]
	\frametitle{Adjazenzliste}
	\begin{wrapfigure}{r}{7cm}
		\vspace{-1cm}
			\begin{tikzpicture}[auto,node distance=0.5cm, semithick]
				\begin{dot2tex}[styleonly,codeonly, fdp]
					digraph G {
						d2ttikzedgelabels = true;
						node [style="state"];
						edge [];
						0->1
						0->3
						2->1 [topath="bend left"]
						2->2 [topath="loop above"]
						3->2
						2->3
					}
				\end{dot2tex}
			\end{tikzpicture}
	\end{wrapfigure}
	Man listet für jeden Knoten dessen benachbarte Knoten auf:\\
	\vspace{.5cm}
	\begin{tabular}{cl}
		\toprule
		$0$&\visible<2->{$\{1, 3\}$}\\
		\midrule
		$1$&\visible<3->{$\{\}$}\\
		\midrule
		$2$&\visible<4->{$\{1,2,3\}$}\\
		\midrule
		$3$&\visible<5->{$\{2\}$}\\
		\bottomrule
	\end{tabular}
\end{frame}

\begin{frame}[fragile]
	\frametitle{Adjazenzmatrix}
	\begin{wrapfigure}{r}{7cm}
		\vspace{-2cm}
		\begin{minipage}{7cm}
			\begin{tikzpicture}[auto,node distance=0.5cm, semithick]
				\begin{dot2tex}[styleonly,codeonly, fdp]
					digraph G {
						d2ttikzedgelabels = true;
						node [style="state"];
						edge [];
						0->1
						0->3
						2->1 [topath="bend left"]
						2->2 [topath="loop above"]
						3->2
						2->3
					}
				\end{dot2tex}
			\end{tikzpicture}
		\end{minipage}
	\end{wrapfigure}
	\begin{tabular}{c||cccc}
		\toprule
		&0&1&2&3\\
		\toprule
		0&\pause0&1&0&1\\
		\midrule
		1&\pause0&0&0&0\\
		\midrule
		2&\pause0&1&1&1\\
		\midrule
		3&\pause0&0&1&0\\
		\bottomrule
	\end{tabular}\\
	\pause
	\vspace{.5cm}
	$A=\begin{pmatrix}0&1&0&1\\0&0&0&0\\0&1&1&1\\0&0&1&0\end{pmatrix}$
\end{frame}
\begin{frame}
	\frametitle{Adjazenzmatrix}
	\begin{definition}
		$A_{ij} = \begin{cases}1&\text{falls } (i,j)\in E\\0&\text{falls } (i,j)\notin E\end{cases}$
	\end{definition}
\end{frame}

\subsection{Anwendung}
\begin{frame}
\frametitle{Liste vs. Matrix}
	\begin{block}{Adjazenzlisten}
	\begin{itemize}
		\visible<2->{\item einfacher Zugriff auf alle adjazenten Knoten}
		\visible<3->{\item um zu überprüfen, ob eine Kante existiert, muss man eventuell alle Nachbarn durchgehen}
	\end{itemize}
	\end{block}

	\begin{block}{Adjanzenzmatrixen}
	\begin{itemize}
		\visible<4->{\item schnelle Überprüfung, ob eine Kante zwischen zwei Knoten $i$ und $j$ existiert}
		\visible<5->{\item um auf einen Nachbarn zuzugreifen, muss man eventuell alle Knoten durchgehen}
	\end{itemize}
	\end{block}
\end{frame}
\begin{frame}
\frametitle{Liste vs. Matrix}
	\begin{block}{Welche Darstellungsform ist geeigneter?}
		Für einen...
		\begin{itemize}
			\visible<1->{\item vollständigen Graphen? \visible<2->{\\\emph{Adjazenzmatrix}}}\pause
			\item Graphen mit nur wenigen Kanten? \visible<3->{\\\emph{Adjazenzliste}}\pause
		\end{itemize}
	\end{block}
\end{frame}

