\section{Logik}
\subsection{Operatoren}
\begin{frame}
  \frametitle{Primitive Operatoren}
  \begin{definition}
    Seien A und B \emph{Aussagen}. w: Wahr, f: Falsch
    
  	\begin{table}
    	\begin{tabular}{|l|l||c||c|}
    	\hline
    	A & B & $ A \wedge B$ & $A \vee B$\\
      \hline
	      f & f & f & f \\
	      f & w & f & w \\
	      w & f & f & w \\
	      w & w & w & w \\
      \hline
      \end{tabular}
    	\begin{tabular}{|l||c|}
    	\hline
    	A & $\neg A$\\
      \hline
	      f & w\\
	      w & f\\
      \hline

      \end{tabular}
       \caption{Und: $\wedge$, Oder: $\vee$, Nicht: $\neg$}
    \end{table}
  \end{definition}
\end{frame}
\begin{frame}
  \frametitle{Implikation}
  \begin{definition}
    \begin{description}
    \item["'Wenn A gilt dann gilt auch B"'] $(A \Rightarrow B) :\Leftrightarrow \neg (A \wedge \neg B)$
    \item["'Wenn A gilt \emph{genau} dann gilt auch B"']$(A \Leftrightarrow B) :\Leftrightarrow ((A \Rightarrow B) \wedge (A \Leftarrow B))$
    \end{description}
  	\begin{table}
    	\begin{tabular}{|l|l||c|}
    	\hline
    	A & B & $ A \Rightarrow B$\\
      \hline
	      f & f & w \\
	      f & w & w \\
	      w & f & f \\
	      w & w & w \\
      \hline
      \end{tabular}
    \end{table}
  \end{definition}
  \begin{alertblock}{}
    Wenn Aussage A falsch ist die Implikation \emph{immer} wahr.
  \end{alertblock}
\end{frame}

\subsection{Beispiele}
\begin{frame}
  \frametitle{Beispiele}
  \begin{exampleblock}{}
  	\begin{table}
    	\begin{tabular}{|l|l|c|c||c|}
    	\hline
    	A & B & $A \vee B$ & $A \wedge B$ & $ (A \vee B) \Rightarrow (A \wedge B)$\\
      \hline
	      f & f & \hiddencell{2}{0} & \hiddencell{3}{0} & \hiddencell{4}{1}\\
	      f & w & \hiddencell{2}{1} & \hiddencell{3}{0} & \hiddencell{4}{0}\\
	      w & f & \hiddencell{2}{1} & \hiddencell{3}{0} & \hiddencell{4}{0}\\
	      w & w & \hiddencell{2}{1} & \hiddencell{3}{1} & \hiddencell{4}{1}\\
      \hline
      \end{tabular}
    \end{table}
  \end{exampleblock}
\end{frame}
\begin{frame}
  \frametitle{Praktische Regeln}
  \begin{theorem}{De Morgan'sche Gesetze}
    \begin{enumerate}
      \item $\neg (A \vee B)$ äquivalent zu $\neg A \wedge \neg B$
      \item $\neg (A \wedge B)$ äquivalent zu $ \neg A \vee \neg B$
    \end{enumerate}
  \end{theorem}
  \begin{theorem}{Implikation von Inversen}
    $(\neg A \Rightarrow \neg B) \Leftrightarrow (B \Rightarrow A)$
  \end{theorem}
\end{frame}
