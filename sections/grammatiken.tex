\section{Kontextfreie Grammatiken}
\subsection{Definitionen}
\begin{frame}
  \frametitle{Kontextfreie Grammatiken}
  \begin{definition}
    Seien folgende Mengen gegeben:
    \begin{description}
      \item[N:] Menge von Nichtterminalsymbolen
      \item[T:] Menge von Terminalsymbolen
      \item[S:] $S \in N$ Startsymbol
      \item[$P \subset N \times V^*$:] Menge von Produktionen, \\
      $V := (N \cup T)$
    \end{description}
    Dann heißt der Tupel G := (N, T, S, P) \emph{kontextfreie Grammatik}.
  \end{definition}\pause
  Schreibweisen:
  \begin{itemize}
    \item Anstelle von $(a, b) \in P$ schreibt man $a \longrightarrow b$.
    \item Die Sprache die von G erzeugt wird nennt man L(G).
  \end{itemize}
\end{frame}

\begin{frame}
  \frametitle{Beispiele}
  \begin{exampleblock}{Beispiele}
    \begin{itemize}
      \item $G_1 := (\{X\}, \{a, b\}, X, \{X \longrightarrow aXb | \varepsilon\})$
      \item $G_2 := (\{X, Y, Z\}, \{0, 1\}, X, \{X \longrightarrow 0Y | 1Z, Y \longrightarrow 0Y|\varepsilon, Z \longrightarrow 1Z|\varepsilon\})$
       \item $G_3 := (\{X\}, \{0\}, X, \{X \longrightarrow X | \varepsilon\})$
    \end{itemize}
  \end{exampleblock}
\end{frame}

\begin{frame}
  \frametitle{Ableitungen}
  \begin{definition}
    Sei $G = (T, N, S, P)$ eine Grammatik. $w, w' \in (T \cup N)^*$.
    Man schreibt:
    \begin{description}
      \item[$w \Longrightarrow w'$] Das Wort w kann in das Wort w' durch anwenden
        \emph{einer} Produktion auf ein Nicht-Terminalsymbol abgeleitet werden.
      \item[$w \Longrightarrow^i w'$] Das Wort w kann in das Wort w' durch anwenden von
        \emph{i} Produktionen auf Nicht-Terminalsymbole abgeleitet werden.
      \item[$w \Longrightarrow^* w'$] Das Wort w kann in das Wort w' durch anwenden von
        \emph{beliebig vielen} Produktionen auf Nicht-Terminalsymbole abgeleitet werden.
    \end{description}
  \end{definition}
\end{frame}

\begin{frame}
  \frametitle{Beispiele}
    $G = (\{X, Y\}, \{a, b\}, X, \{X \longrightarrow XY | a, Y \longrightarrow b\})$
  \begin{exampleblock}{}
    \begin{enumerate}
      \item $aX \Longrightarrow aXY$ aber \emph{nicht} $aX \longrightarrow aXY$
      \item $aXbbb \Longrightarrow^2 aaYbbb$ da $aXbbb \Longrightarrow aXYbbb \Longrightarrow aaYbbb$
      \item $ X \Longrightarrow^* aaaaabbbb$
    \end{enumerate}
  \end{exampleblock}
\end{frame}

\subsection{Aufgaben}
\begin{frame}
  \frametitle{Fragen}
  \begin{exampleblock}{}
    \begin{enumerate}
      \item Gibt es Grammatiken für die gilt: $L(G) = \{\}$?
      \item Welche Sprache erzeugt: $G_1 := (\{X\}, \{0\}, X, \{X \longrightarrow X\})$
      \item Ist $G_2 := (\{X\}, \{a, b\}, X, \{X \longrightarrow \varepsilon\})$ eine gültige Grammatik?
    \end{enumerate}
  \end{exampleblock}
  \begin{exampleblock}{}
    \begin{enumerate}
      \pause
      \item Ja z.B. $G = (\{X\}, \{0\}, X, \{\})$
      \pause
      \item $L(G_1) = \{\}$
      \pause
      \item Ja und $L(G_2) = \{\varepsilon\}$
    \end{enumerate}
  \end{exampleblock}
\end{frame}
\begin{frame}
  \frametitle{Aufgaben}
  \begin{exampleblock}{Welche Sprachen erzeugen folgende Grammatiken.}
    \begin{enumerate}
      \item $G_1 := (\{X, Y\}, \{a, b\}, X, \{X \longrightarrow aY | \varepsilon, Y \longrightarrow bX\})$
      \item $G_2 := (\{X, Y, Z\}, \{a, b, c\}, X,$\\
             $\{X \longrightarrow Ya | Yb | Yc, Y \longrightarrow ZZY | \varepsilon, Z \longrightarrow a | b | c\})$
    \end{enumerate}
  \end{exampleblock}
  \begin{exampleblock}{Gebt eine jeweils Grammatik an für die gilt $L(G) = L_i$:}
        \begin{enumerate}
      \item $L_1 := \{ab, cd\}^+ \cdot \{a, c\}^2$
      \item $A := \{0, 1\}$, $L_2 := \{w  \in A^*| Num_0(w) = Num_1(w)\}$
    \end{enumerate}
  \end{exampleblock}
\end{frame}
\begin{frame}
  \frametitle{Bonus-Aufgabe}
  \begin{exampleblock}{In Mengen M aus Studenten mit $|M| \leq 3$}
    Konstruiert eine Grammatik die alle E-Mail-Adresse aus den Buchstaben {a, b, c} erzeugt.
    \emph{Hinweis}: $T := \{a, b, c, @, ., \_\}$
  \end{exampleblock}\pause
  \begin{exampleblock}{Lösung}
    $G = (N, T, S, P)$
    \begin{itemize}
      \item $N = \{E, A, B\}$
      \item $T = \{a, b, c, ., \_, @\}$
      \item $S = E$
      \item $P = \{E \longrightarrow A@B.B, A \longrightarrow BA|\_A|.A, B \longrightarrow aB | bB | cB | \varepsilon\}$
    \end{itemize}
  \end{exampleblock}
\end{frame}
