\section{Vollständige Induktion}
\subsection{Definition}
\begin{frame}
  \frametitle{Vollständige Induktion}
  \begin{definition}
    Sei A(n) ein Aussage die für alle $n \in \mathbb{N}$ gelten soll.
    \begin{enumerate}
      \item IA: Zeige es gilt A(1).
      \item IV: Nehme an das A(n) für ein bliebiges aber festes $n \in \mathbb{N}$ gilt.
      \item IS: Zeige mit (IV) das A(n+1) gilt.
    \end{enumerate}
  \end{definition}
  \begin{alertblock}{Sehr wichtig!}
    Kommt fast immer in der Klausur.
  \end{alertblock}
\end{frame}
\begin{frame}[plain]
  \frametitle{}
  \begin{exampleblock}{Gaußsche Summenformel}
    \emph{Behauptung:} $\forall n \in \mathbb{N}: \sum \limits^{n}_{i=1} i = \frac{n(n+1)}{2}$\pause\\
    \emph{Beweis}:
    \begin{description}{}
      {\tiny
      \item[IA:] $n = 1$: $\sum \limits^{1}_{i=1} i = 1 = \frac{1(1+1)}{2}$
      \item[IV:] Für ein bliebiges aber festes $n \in \mathbb{N}$ gelte: $\sum \limits^{n}_{i=1} i = \frac{n(n+1)}{2}$
      \item[IS:] $n \leadsto n+1$:
        \begin{align*}
          \sum \limits^{n+1}_{i=1} i &= \sum \limits^{n}_{i=1} i + \sum \limits^{n+1}_{i=n} i
          = \sum \limits^{n}_{i=1} i + n+1\\
          &= \frac{n(n+1)}{2} + n+1 = \frac{n(n+1) + 2(n+1)}{2}\\
          &= \frac{(n+2) \cdot (n+1)}{2} = \frac{(n+1)((n+1)+1)}{2} \blacksquare
        \end{align*}
       }
    \end{description}
  \end{exampleblock}
\end{frame}

\subsection{Aufgaben}
\begin{frame}
  \frametitle{Aufgaben}
  \begin{exampleblock}{In Mengen M aus Studenten mit $|M| \leq 3$}
    Beweist folgende Aussagen mit vollständiger Induktion.
    \begin{enumerate}
      \item $\forall n \in \mathbb{N}: \sum \limits^{n}_{k=1}(2k-1) = n^2$
      \item $\forall n \in \mathbb{N}: n(n^2-1)$ ist durch 3 (restlos) teilbar.\\
        {\tiny Tipp: Wie würdet ihr Teilbarkeit definieren?}
    \end{enumerate}
  \end{exampleblock}
\end{frame}