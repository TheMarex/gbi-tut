\section{Übersetzungen}
\subsection{Warum?}
\begin{frame}
	\frametitle{Wozu Übersetzungen?}
	\pause
	\begin{itemize}
		\item Lesbarkeit
		\item Kompression
		\item Verschlüsselung
		\item Fehlererkennung und -korrektur
		\item \dots
	\end{itemize}
\end{frame}

\subsection{Homomorphismen}
\begin{frame}
	\begin{exampleblock}{Beispiel}
		$h(a)=001$\\
		$h(b)=1101$\\
		Also gilt:
		\begin{eqnarray*}
			h(bba)&=&h(b)h(b)h(a)\\
			&=&1101\cdot 1101\cdot 001\\
			&=&11011101001
		\end{eqnarray*}
	\end{exampleblock}
	\pause
	\begin{definition}
		Der Homomorphismus $h^{**}:A^*\rightarrow B^*$ sei wie folgt definiert:
		\begin{eqnarray*}
			h^{**}(\varepsilon)&=&\varepsilon \\
			\forall w\in A^*:\forall x\in A:h^{**}(wx)&=&h^{**}(w)h(x)
		\end{eqnarray*}
	\end{definition}
\end{frame}

\subsection{$\varepsilon$-freier Homomorphismus}
\begin{frame}
	\frametitle{$\varepsilon$-freier Homomorphismus}
	\begin{definition}
		Ein Homomorphismus ist $\varepsilon$-frei, wenn für alle $x\in A$ gilt: $h(x)\neq \varepsilon$.
	\end{definition}

	Warum sollte ein Homomorphismus $\varepsilon$-frei sein?
	\begin{exampleblock}{Beispiel}
		$h(a)=001$\\
		$h(b)=\varepsilon$\\
		\pause
		Es sei $h(w)=001$. Dann ist $w$ nicht eindeutig bestimmt.
		Es ist "`Information verloren gegangen"'.
	\end{exampleblock}
\end{frame}
\begin{frame}
	\frametitle{"`Informationsverlust"'}
	\begin{exampleblock}{Beispiel}
		$h(a)=0$\\
		$h(b)=1$\\
		$h(c)=10$\\
		\visible<2->{$h(bac)=1010$}\visible<3->{$=h(cab)=h(cc)$}
	\end{exampleblock}
	\begin{itemize}
		\item Warum geht bei dieser Übersetzung Information verloren?
		\pause
		\item Wie kann man allgemein ausdrücken, wann Information verloren geht?\\
		\visible<3->{$\exists w_1\neq w_2$ mit $h^{**}(w_1)=h^{**}(w_2)$}
	\end{itemize}
	\pause
\end{frame}
\subsection{Präfixfreiheit}
\begin{frame}
	\frametitle{Präfixfreiheit}
	\begin{definition}
		Für keine zwei Symbole $x_1,x_2\in A$ gilt: $h(x_1)$ ist Präfix von $h(x_2)$
	\end{definition}
	\begin{exampleblock}{Beispiele}
		Welche Codierungen sind präfixfrei?:
		\begin{itemize}
			\item $h(a)=001, h(b)=1101 $ \pause \checkmark\\
			\item $h(a)=01, h(b)=011 $ \pause x\\
			\item $h(a)=011, h(b)=110 $ \pause \checkmark\\
			\item $h(a)=42, h(b)=4, h(b)=24 $ \pause x\\
		\end{itemize}

		
	\end{exampleblock}

\end{frame}




