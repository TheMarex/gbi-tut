\section{Strukturelle Induktion}

\subsection*{}
\begin{frame}
  \frametitle{Motivation}
  Wir wollen die vollständige Induktion "verallgemeinern".
  Vorgehen:
  \begin{enumerate}
    \item Aussage für "grundlegende Elemente" beweisen.
    \item Beweisen das die Aussage auch für daraus zusammen gesetzte Elemente gilt.
  \end{enumerate}
\end{frame}
\begin{frame}
  \frametitle{Beispiel}
  Die Menge $M \subseteq \mathbb{Z}^2$ sei wie folgt definiert:
  \begin{itemize}
    \item $(3, 2) \in M$
    \item Wenn $(m, n) \in M$, dann ist auch $(3m - 2n, m) \in M$
    \item Keine weiteren Elemente liegen in $M$.
  \end{itemize}
  \begin{exampleblock}{Aufgabe}
    Zeigen Sie: $\forall m \in M: \exists k \in \mathbb{N}_0: m = (2^{k+1}+1, 2^k + 1)$
  \end{exampleblock}
\end{frame}
