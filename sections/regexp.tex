\section{Reguläre Ausdrücke}
\subsection{Definitionen}
\begin{frame}
  \frametitle{Definition}
  \emph{Reguläre Ausdrücke} sind eine verbreitete Notation, um reguläre
  Sprachen (Typ-3) zu beschreiben.
	\begin{block}{Syntax}
		\begin{tabular}{c p{0.7\textwidth}}
				\xb{Zeichen} & 	\xb{Bedeutung} \\
				$( )$			& Klammerung von Alternativen\\
				$\cdot$			& Verkettet Ausdrücke\\
				$|$			& trennt Alternativen\\
				$*$			& beliebig häufiges Vorkommen\\
		\end{tabular}
	\end{block}
  \begin{itemize}
      \item $*$ bindet stärker als Verkettung
      \item Verkettung $(R \cdot S)$ bindet stärker als "`oder"' $(R|S)$
  \end{itemize}
\end{frame}

\begin{frame}
  \frametitle{Die Sprache von R}
	\begin{definition}
		Sei $R$ ein regulärer Ausdruck, dann bezeichnet $\langle R \rangle$ die von ihm erzeugte Sprache.
    Es gilt:
    \begin{itemize}
      \item $\langle \emptyset \rangle =\{\}$
      \item Für $a \in A$ ist $\langle a \rangle=\{a\}$
      \item $\langle R_1 | R_2 \rangle = \langle R_1 \rangle \cup \langle R_2 \rangle$
      \item $\langle R_1 R_2 \rangle = \langle R_1 \rangle \cdot \langle R_2 \rangle$
    \end{itemize}
    $R_1$ und $R_2$ sind hier zwei beliebige reguläre Ausdrücke.
	\end{definition}
\end{frame}

\begin{frame}
  \frametitle{Beispiele}
	\begin{exampleblock}{}
    \begin{itemize}
      \item $R_1 = ({(a|b)*}c)|({(c|b)*}a)$
      \item $R_2 = {(a|b|c|\dots|z)*}.jpg$
      \item $R_3 = {(a|b|c|\dots|z)*}@{(a|b|c|\dots|z)*}.{(a|b|c|\dots|z)*}$
      \item $Z := (a|b|c|\dots|z)$\\
            $R_4 = \text{http://}{Z*}.{Z*}(\emptyset|/Z*)*$
    \end{itemize}
	\end{exampleblock}
\end{frame}

\subsection{Aufgaben}
\begin{frame}
  \frametitle{Aufgaben}
  \begin{enumerate}
		\item Welche Wörter erzeugt der reguläre Ausdruck: $R=(a|b)*abb(a|b)*$\\
    \ans{2}{$\langle R \rangle$ enthält alle Wörter mit dem Teilwort $abb$}
		\item Gebe einen regulären Ausdruck für die Sprache aller Wörter die nicht $ab$ enthalten\\
		\ans{2}{$R = {b*} {a*}$}
  \end{enumerate}
\end{frame}

\begin{frame}
  \frametitle{Aufgaben}
  \begin{enumerate}
		\item Welcher reguläre Ausdruck R erzeugt die Sprache $L=\{\epsilon\}$?
      \ans{2}{$\emptyset*$, denn $\langle \emptyset \rangle ^* = \{\}^* = \{\epsilon\}$}
		\item Gebe einen regulären Ausdruck für die Sprache aller Wörter mit mindestens 3 b's an!\\
			\ans{2}{${(a|b)*}{b(a|b)*}{b(a|b)*}{b(a|b)*}$ oder ${a*}{ba*}{ba*}b{(a|b)*}$}
	\end{enumerate}
\end{frame}

\begin{frame}
  \frametitle{Aufgabe}
  Gegeben ist folgende Klasse von Sprachen über dem Alphabet $\Sigma = \{a,b,c\}$:
  \begin{multline*}
    L_n = \{w|
      \text{w enthält genau einmal das Teilwort $w'=a^n$}\\
      \text{das nicht Teil von $a^k$ mit $k > n$ ist.}
          \}
  \end{multline*}
  Gebt einen regulären Ausdruck $R$ für $L_4$ an! (also: $\langle R \rangle = L_4$)\\
  \hfill
  \ans{2}{
    Wir stellen sicher, dass $aaaa$ genau einmal vorkommt, und sonst nur 1, 2, 3 oder mehr $a$'s.
    \begin{multline*}
      R = ( (b|c)* (\emptyset |a|aa|aaa|aaaaaa*)(b|c)(b|c)*)* \\ aaaa (
      (b|c)(b|c)*(\emptyset |a|aa|aaa|aaaaaa* )(b|c)*)*
    \end{multline*}}
\end{frame}

\begin{frame}
  \frametitle{Aufgabe}
	\begin{exampleblock}{}
		Sei $R$ ein regulärer Ausdruck für eine formale Sprache $L=\langle R \rangle$. Konstruiere einen regulärer Ausdruck
  		\begin{enumerate}
        \item für $L^*$ \\
          \ans{2}{$R_* = (R)*$}
        \item für $L^+$ \\
          \ans{2}{$R_+ = R(R)*$}
  		\end{enumerate}
	\end{exampleblock}
\end{frame}

