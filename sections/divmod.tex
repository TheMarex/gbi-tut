\section{Ganzzahlige Division mit Rest}
\subsection{Definition}
\begin{frame}
  \frametitle{Ganzzahlige Devision mit Rest}
  \begin{definition}
    Sei $z \in \mathbb{Z}$ eine ganze Zahl. Dann kann man für jede Zahl $n \in \mathbb{Z}$ \emph{eindeutige} Zahlen $p, r \in \mathbb{Z}$ finden, so das gilt: $z = p \cdot n + r$ und man definiert:
    \begin{description}
      \item $z \:\textnormal{mod}\: n := r$
      \item $z \:\textnormal{div}\: n := p$
    \end{description}
  \end{definition}
  \begin{alertblock}{In einfachen Worten:}
   $z \:\textnormal{mod}\: n := r$ entspricht dem Rest einer ganzzahligen Division.\\
   $z \:\textnormal{div}\: n := p$ entspricht dem Quotient einer ganzzahligen Division.\\
   \emph{Denkt an die schriftliche Division aus der Grundschule.}
  \end{alertblock}
\end{frame}

\subsection{Beispiele}
\begin{frame}
  \frametitle{Beispiele}
  \begin{exampleblock}{Jetzt seid ihr gefragt.}
  	\begin{table}
    	\begin{tabular}{r||c|c|c|c|l}
    	x & 3 & 5 & 12 & 4 & 17\\
    	\hline
    	\hline
	    x mod 5  & \hiddencell{2}{3} & \hiddencell{2}{0} & \hiddencell{2}{2} & \hiddencell{2}{4} & \hiddencell{2}{2} \\
	    x div 5  & \hiddencell{3}{0} & \hiddencell{3}{1} & \hiddencell{3}{2} & \hiddencell{3}{0} & \hiddencell{3}{3} \\
	    2 $\cdot$ (x div 2)  & \hiddencell{4}{2} & \hiddencell{4}{4} & \hiddencell{4}{12} & \hiddencell{4}{2} & \hiddencell{4}{16} \\
      \end{tabular}
    \end{table}
  \end{exampleblock}
\end{frame}
