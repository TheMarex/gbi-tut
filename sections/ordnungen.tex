\section{Ordnungen}
\begin{frame}
	\frametitle{Ordnungen}
	\begin{definition}
		Eine \emph{Ordnung} ist eine Halbordnung für die zusätzlich gilt:
		$$\forall x,y\in M: xRy \vee yRx$$
	\end{definition}
\end{frame}
\begin{frame}
	\frametitle{Beispiele}
	\begin{exampleblock}{Lexikographische Ordnung I}
		Die Relation \emph{$\sqsubseteq_1$} sei die Ordnungsrelation, nach der z.B. die Wörter in einem Wörterbuch geordnet sind.\\
		Ergänze:
		\begin{itemize}
			\item aa \visible<2->{$\sqsubseteq_1$} aabba
			\item bba \visible<3->{$\sqsupseteq_1$} aa
			\item aaaaa \visible<4->{$\sqsubseteq_1$} bba
			\item aab \visible<5->{$\sqsupseteq_1$} aaaab
		\end{itemize}
	\end{exampleblock}
\end{frame}
\begin{frame}
	\begin{exampleblock}{Lexikographische Ordnung II}
		Die Relation \emph{$\sqsubseteq_2$} sei die Ordnungsrelation, die zuerst nach Länge und danach alle gleich langen Wörter wie im Wörterbuch ordnet.\\
		Ergänze:
		\begin{itemize}
			\item aa \visible<2->{$\sqsubseteq_1$} aabba
			\item bba \visible<3->{$\sqsupseteq_1$} aa
			\item aaaaa \visible<4->{$\sqsupseteq_1$} bba
			\item aab \visible<5->{$\sqsubseteq_1$} aaaab
		\end{itemize}
	\end{exampleblock}
\end{frame}