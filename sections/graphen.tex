\section{Graphen}
\subsection{Definitionen}
\begin{frame}
  \frametitle{Gerichtete Graphen}
  \begin{definition}
    \begin{description}{Seien folgende Mengen gegeben:}
      \item[V:] Menge von Knoten
      \item[$E \subset V \times V$:] Menge von Kanten
    \end{description}
    Dann heißt der Tupel G := (V, E) ein gerichteter \emph{Graph}.
  \end{definition}\pause
  \begin{exampleblock}{Beispiele}
    \begin{itemize}
    \end{itemize}
  \end{exampleblock}
\end{frame}
\begin{frame}
  \frametitle{Knotengrad}
  \begin{definition}
    Sei G = (V, E) ein gerichteter Graph.
    \begin{description}
      \item[Ausgangsgrad:] $d_+(x) := |{(x, e) \in E | e \in V}|$
      \item[Eingangsgrad:] $d_-(x) := |{(e, x) \in E | e \in V}|$
    \end{description}
  \end{definition}\pause
\end{frame}

\subsection{Aufgaben}
\begin{frame}
  \frametitle{Aufgaben}
  \begin{exampleblock}{In Mengen M aus Studenten mit $|M| \leq 3$}
      Anweisung
      \begin{enumerate}
        \item Aufgabe
      \end{enumerate}
  \end{exampleblock}
\end{frame}
