
\section{Motivation Graphen}
\begin{frame}[fragile]
  \frametitle{Was ist ein Graph?}
  \begin{figure}
    \begin{minipage}{5cm}
      \begin{tikzpicture}[->,>=stealth',shorten >=1pt,auto,node distance=2.8cm,
                          semithick]
        \begin{dot2tex}[styleonly,codeonly,neato]
          digraph G {
            d2ttikzedgelabels = true;
            node [style="state"];
            edge [style="-to",topath="bend left"];
            A -> B
            B -> B [topath="loop above"]
            B -> C
            C -> B
            C -> A
          }
        \end{dot2tex}
      \end{tikzpicture}
    \end{minipage}
    \begin{minipage}{5cm}
      \begin{tikzpicture}[auto,node distance=2.0cm, semithick]
        \begin{dot2tex}[styleonly,codeonly,neato]
          graph G {
            d2ttikzedgelabels = true;
            node [style="state"];
            edge [];
            A -- B
            B -- C
            B -- D
          }
        \end{dot2tex}
      \end{tikzpicture}
    \end{minipage}
  \end{figure}
\end{frame}

\section{Gerichtete Graphen}
\subsection{Definitionen}
\begin{frame}
  \frametitle{Gerichtete Graphen}
  \begin{definition}
    \begin{description}
      \item[V:] Menge von Knoten
      \item[$E \subseteq V \times V$:] Menge von Kanten
    \end{description}
    Dann heißt der Tupel G := (V, E) ein gerichteter \emph{Graph}.
  \end{definition}
\end{frame}
\begin{frame}
	\frametitle{Wichtige Begriffe}
	\begin{definition}
    Sei $G = (V, E)$ ein Graph und $x, y \in V$.
		\begin{description}
			\item[x und y sind adjazent:] \hfill \\
        Es gibt eine Kante $(x,y)\in E$.\pause
			\item[Schlinge:] \hfill \\
        Eine Kante der Form $(x,x)\in E$.\pause
			\item[Schlingenfrei:] \hfill \\
        Ein Graph besitzt keine Schlingen.\pause
			\item[Teilgraph:] \hfill \\
        Ein Graph $G'=(V',E')$ mit $V'\subseteq V$ und $E'\subseteq E \cap V'\times V'$.
		\end{description}
	\end{definition}
\end{frame}
\begin{frame}[fragile]
  \frametitle{Beispiele}
  \begin{exampleblock}{}
    \begin{enumerate}
      \item $G_1 := (\{0, 1, 2, 3\}, \{(0, 0), (0, 1), (1, 2), (2, 1)\})$
      \item $G_2 := (\{0, 1, 2, 3\}, \{(0, 2), (3, 1), (2, 1)\})$
    \end{enumerate}
  \end{exampleblock}
  \begin{figure}
    \begin{minipage}{5cm}
      \begin{tikzpicture}[->,>=stealth',shorten >=1pt, auto, semithick, scale=.50]
        \begin{dot2tex}[styleonly,codeonly,neato]
          digraph G {
            d2ttikzedgelabels = true;
            node [style="state"];
            edge [style="-to",topath="bend left"];
            1 -> 2
            2 -> 1;
            0 -> 1
            0 -> 0 [topath="loop below"]
            3;
          }
        \end{dot2tex}
      \end{tikzpicture}
      \caption{$G_1$}
    \end{minipage}
    \begin{minipage}{5cm}
      \begin{tikzpicture}[->,>=stealth',shorten >=1pt, auto, semithick, scale=.75]
        \begin{dot2tex}[styleonly,codeonly,neato]
          digraph G {
            d2ttikzedgelabels = true;
            node [style="state"];
            edge [style="-to",topath="bend left"];
            0 -> 2
            3 -> 2
            1 -> 2
            1 -> 3
          }
        \end{dot2tex}
      \end{tikzpicture}
      \caption{$G_2$}
    \end{minipage}
  \end{figure}
\end{frame}

\subsection{Knotengrad}
\begin{frame}
  \frametitle{Knotengrad}
  \begin{definition}
    Sei G = (V, E) ein gerichteter Graph.
    \begin{description}
      \item[Ausgangsgrad:] $d_+(x) := |\{(x, e) \in E | e \in V\}|$
      \item[Eingangsgrad:] $d_-(x) := |\{(e, x) \in E | e \in V\}|$
    \end{description}
  \end{definition}
\end{frame}
\begin{frame}[fragile]
  \frametitle{Welchen Grad haben die Knoten?}
    \begin{figure}
      \begin{tikzpicture}[->,>=stealth',shorten >=1pt, auto, semithick, scale=.75]
        \begin{dot2tex}[styleonly,codeonly,neato]
          digraph G {
            d2ttikzedgelabels = true;
            node [style="state"];
            edge [style="-to",topath="bend left"];
            0 -> 2
            3 -> 2
            1 -> 2
            1 -> 3
          }
        \end{dot2tex}
      \end{tikzpicture}
    \end{figure}
\end{frame}

\subsection{Pfade}
\begin{frame}
	\frametitle{Pfade}
	\begin{block}{Definition}
		Ein Pfad ist eine Liste $p=(v_0, \ldots ,v_n) \in V^+$ mit: $\forall i \in {\mathbb G}_n: (v_i,v_{i+1})\in E$
		\begin{description}
			\item[Länge:]\hfill \\
        Die Anzahl der Kanten $n = |p| - 1$. \pause
			\item[Wiederholungsfrei:] \hfill \\
        Alle Knoten sind je paarweise verschieden, maximal $v_0$ und $v_n$ sind gleich. \pause
			\item[Geschlossen:] \hfill \\
        Wenn gilt $v_0 = v_n$. Der Pfad heißt dann auch \emph{Zyklus}.\pause
			\item[Einfacher Zyklus:] \hfill \\
        Ein geschlossener und wiederholungsfreier Pfad.
    \end{description}
	\end{block}
\end{frame}
\begin{frame}[fragile]
  \frametitle{Welche Pfade gibt es in diesem Graphen?}
    \begin{figure}
      \begin{tikzpicture}[->,>=stealth',shorten >=1pt, auto, semithick, scale=.75]
        \begin{dot2tex}[styleonly,codeonly,neato]
          digraph G {
            d2ttikzedgelabels = true;
            node [style="state"];
            edge [style="-to",topath="bend left"];
            1 -> 2
            2 -> 1;
            0 -> 1
            0 -> 0 [topath="loop below"]
            3;
          }
        \end{dot2tex}
      \end{tikzpicture}
    \end{figure}
\end{frame}

\section{Gerichtete Bäume}
\subsection{Definition}
\begin{frame}
	\frametitle{Gerichtete Bäume}
	\begin{definition}
    Ein gerichteter Graph $G = (V, E)$ heißt \emph{gerichteter Baum} wenn gilt:
		\begin{enumerate}
			\item $\exists r \in V$ mit: $\forall x\in V$ ex. genau ein Pfad von $r$ nach $x$ \pause
			\item Es gibt genau einen solchen Knoten $r$.
		\end{enumerate}
	\end{definition}
\end{frame}
\begin{frame}[fragile]
  \frametitle{Beispiel}
    \begin{figure}
      \begin{tikzpicture}[->,>=stealth',shorten >=1pt, auto, semithick, scale=.75]
        \begin{dot2tex}[styleonly,codeonly,dot]
          digraph G {
            d2ttikzedgelabels = true;
            node [style="state"];
            edge [style="-to"];
            5 -> 3;
            5 -> 7;
            3 -> 2;
            3 -> 4;
            7 -> 6;
            7 -> 9;
          }
        \end{dot2tex}
      \end{tikzpicture}
    \end{figure}
\end{frame}

\section{Ungerichtete Graphen}
\subsection{Definition}
\begin{frame}
  \frametitle{Ungerichtete Graphen}
  \begin{definition}
    \begin{description}
      \item[V:] Menge von Knoten
      \item[$E \subseteq \{\{x, y\} | x, y \in V\}$:] Menge von Kanten
    \end{description}
    Dann heißt der Tupel $U := (V, E)$ ein ungerichteter \emph{Graph}.\\
    Wir nennen $G_U = (V, \{(x, y) | \{x, y\} \in E\})$ den zu U gehörenden gerichteten Graphen.
  \end{definition}
\end{frame}
\begin{frame}[fragile]
	\frametitle{Beispiel}
  \begin{figure}
    \begin{dot2tex}[straightedges,circo]
      graph G {
        mindist = 0.5;
        node [shape="circle"];
        a -- b;
        a -- c;
        a -- d;
        b -- c;
        d -- d;
        d -- c;
      }
    \end{dot2tex}
  \end{figure}
	\only<1->{Wie sähe dieser ungerichtete Graph als Menge aus?\\}
	\only<2->{$G = ( \{a,b,c,d\}, \{\{a,b\},\{a,c\},\{a,d\},\{b,c\},\{d,b\},\{d,c\}\})$}
\end{frame}

\subsection{Wege}
\begin{frame}
	\frametitle{Wege}
	\begin{definition}
		Ein Weg ist eine Liste $p=(v_0, \ldots ,v_n) \in V^+$ mit: $\forall i \in {\mathbb G}_n: \{v_i,v_{i+1}\} \in E$.\\
    Länge, Wiederholungfreiheit, Geschlossenheit sind analog zu Pfaden definiert.
	\end{definition}
\end{frame}
\begin{frame}[fragile]
  \frametitle{Welche Wege gibt es in diesem Graphen?}
    \begin{figure}
      \begin{tikzpicture}[shorten >=1pt, auto, semithick, scale=.5]
        \begin{dot2tex}[styleonly,codeonly,neato]
          graph G {
            d2ttikzedgelabels = true;
            node [style="state"];
            edge [path="bend left"];
            0 -- 2;
            2 -- 3;
            2 -- 6;
            1 -- 4;
            1 -- 5;
          }
        \end{dot2tex}
      \end{tikzpicture}
    \end{figure}
\end{frame}

\begin{frame}
	\frametitle{Zusammenhängende Graphen}
	\begin{definition}
    Sei $G = (V, E)$ ein Graph.
		\begin{description}
			\item[G ist gerichtet:] \hfill \\
			  G heißt \emph{streng zusammenhängend} wenn für jedes Knotenpaar $(x,y)\in V^2$ gilt: Es gibt einen Pfad von $x$ nach $y$.
			\item[G ist ungerichtet:] \hfill \\
        G heißt \emph{zusammenhängend}, wenn der entsprechende gerichtete Graph \emph{streng zusammenhängend} ist.
		\end{description}
	\end{definition}
\end{frame}

\begin{frame}
	\frametitle{Isomorphe Graphen}
	\begin{definition}
    Seien $G_1 = (V_1, E_1), G_2 = (V_2, E_2)$ zwei Graphen.\\
    Wenn eine Bijektion $f: E_1 \rightarrow E_2$ existiert, sodass gilt:\\
    $\{(f(x), f(y)) | (x, y) \in E_1\} = E_2$ heißt $G_1$ zu $G_2$ \emph{isomorph}.
	\end{definition}
\end{frame}

\subsection{Aufgaben}
\begin{frame}[fragile]
  \frametitle{Aufgaben}
  \begin{exampleblock}{}
      Welche der Graphen sind isomorph? Gebt ggf. eine passende Knoten-Bijektion an.
    \begin{figure}
      \begin{minipage}{4cm}
        \begin{tikzpicture}[auto, semithick, scale=.5]
          \begin{dot2tex}[codeonly,twopi]
            graph G {
              d2ttikzedgelabels = true;
              node [shape="circle"];
              edge [];
              0 -- 1;
              0 -- 2;
              0 -- 3;
              0 -- 4;
              1 -- 2;
              1 -- 4;
              2 -- 3;
              3 -- 4;
            }
          \end{dot2tex}
        \end{tikzpicture}
        \caption{$G_1$}
      \end{minipage}
      \begin{minipage}{3cm}
        \begin{tikzpicture}[auto, semithick, scale=.5]
          \begin{dot2tex}[codeonly,neato]
            graph G {
              d2ttikzedgelabels = true;
              node [shape="circle"];
              edge [];
              0 -- 1
              0 -- 2
              0 -- 3
              0 -- 4;
              1 -- 2 -- 3 -- 4;
            }
          \end{dot2tex}
        \end{tikzpicture}
        \caption{$G_2$}
      \end{minipage}
      \begin{minipage}{3cm}
        \begin{tikzpicture}[auto, semithick, scale=.5]
          \begin{dot2tex}[codeonly,circo]
            graph G {
              d2ttikzedgelabels = true;
              node [shape="circle"];
              edge [];
              2 -- 5;
              2 -- 0;
              2 -- 3;
              2 -- 4;
              5 -- 0;
              5 -- 3;
              0 -- 4;
              3 -- 4;
            }
          \end{dot2tex}
        \end{tikzpicture}
        \caption{$G_2$}
      \end{minipage}
    \end{figure}
  \end{exampleblock}
\end{frame}
