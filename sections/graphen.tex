
\section{Graphen}
\subsection{Motivation}
\begin{frame}[fragile]
  \frametitle{Was ist ein Graph?}
  \begin{tikzpicture}[->,>=stealth',shorten >=1pt,auto,node distance=2.8cm,
                      semithick]
    \begin{dot2tex}[styleonly,codeonly,neato]
      digraph G {
        d2ttikzedgelabels = true;
        node [style="state"];
        edge [style="-to",topath="bend left"];
        A -> B
        B -> B [topath="loop above"]
        B -> C
        C -> B
        C -> A
      }
    \end{dot2tex}
  \end{tikzpicture}
\end{frame}

\subsection{Definitionen}
\begin{frame}[fragile]
  \frametitle{Gerichtete Graphen}
  \begin{definition}
    \begin{description}
      \item[V:] Menge von Knoten
      \item[$E \subset V \times V$:] Menge von Kanten
    \end{description}
    Dann heißt der Tupel G := (V, E) ein gerichteter \emph{Graph}.
  \end{definition}\pause
  \begin{exampleblock}{Beispiele}
  \end{exampleblock}
\end{frame}
\begin{frame}
  \frametitle{Knotengrad}
  \begin{definition}
    Sei G = (V, E) ein gerichteter Graph.
    \begin{description}
      \item[Ausgangsgrad:] $d_+(x) := |\{(x, e) \in E | e \in V\}|$
      \item[Eingangsgrad:] $d_-(x) := |\{(e, x) \in E | e \in V\}|$
    \end{description}
  \end{definition}\pause
\end{frame}

\subsection{Aufgaben}
\begin{frame}
  \frametitle{Aufgaben}
  \begin{exampleblock}{In Mengen M aus Studenten mit $|M| \leq 3$}
      Anweisung
      \begin{enumerate}
        \item Aufgabe
      \end{enumerate}
  \end{exampleblock}
\end{frame}
