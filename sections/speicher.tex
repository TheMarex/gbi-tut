\section{Speicher}
\subsection{Bit}
\begin{frame}
	\frametitle{Bit}
	\begin{description}
		\item[Was ist ein Bit?] \only<2->{Ein Zeichen aus dem Alphabet $\{0,1\}$}
	\end{description}
	\pause
	\pause
	\begin{block}{Herkunft des Namens}
		\begin{itemize}
			\item \textbf{bi}nary dig\textbf{it}
			\item \textbf{B}asic Indissoluble \textbf{I}nformation Uni\textbf{t}
		\end{itemize}
	\end{block}
	\pause
	\begin{block}{Abkürzung}
		$b$ oder $bit$
	\end{block}
\end{frame}

\subsection{Byte}
\begin{frame}
	\frametitle{Byte}
	\begin{description}
		\item[Was ist ein Byte?] Ein Wort über dem Alphabet $\{0,1\}$ der Länge 8.
    \item[Beispiel:] 01000000
  \end{description}
	\pause
	\begin{block}{Herkunft des Namens}
		In Anlehnung an die englischen Wörter \textit{bit} und \textit{bite}.\\
		Oft auch als \textit{Octet} bezeichnet.
	\end{block}
	\pause
	\begin{block}{Abkürzung}
		$B$ für $Byte$\\
		$o$ für $Octet$\\
	\end{block}
\end{frame}

\subsection{Größenpräfixe}
\begin{frame}[plain]
	\frametitle{Speicher-Präfixe}
	Um die Größe von "`großen"' Speichern zu beschreiben verwendet man Präfixe:
	\begin{block}{Dezimal}
		\begin{tabular}{rccc}
			1000&$10^{3}$&kilo&k\\
			1000000&$10^{6}$&mega&M\\
			1000000000&$10^{9}$&giga&G\\
			1000000000000&$10^{12}$&tera&T\\
			1000000000000000&$10^{15}$&peta&P\\
			1000000000000000000&$10^{18}$&exa&E\\
		\end{tabular}

	\end{block}
	\begin{block}{Binär}
		\begin{tabular}{rccc}
			$1024$&$2^{10}$&kibi&ki\\
			$1048576$&$2^{20}$&mebi&Mi\\
			$1073741824$&$2^{30}$&gibi&Gi\\
			$1,099511628*10^{12}$&$2^{40}$&tebi&Ti\\
			$1,125899907*10^{15}$&$2^{50}$&pebi&Pi\\
			$1,152921505*10^{18}$&$2^{60}$&exbi&Ei\\
		\end{tabular}
	\end{block}

\end{frame}


\subsection{Speicher als Abbildung}
\begin{frame}
	\frametitle{Speicher als Abbildung}
	\begin{tabular}{cc}
		\toprule
		Adresse&Wert\\
		\midrule
		000&10101000\\
		001&01101001\\
		010&01010011\\
		011&10100110\\
		100&11001101\\
		101&11011001\\
		110&00101011\\
		111&00000100\\
		\bottomrule
	\end{tabular}
	\begin{block}{Abbildung}
		$m:\text{Adr}\rightarrow\text{Val}$\\
		\vspace{.2cm}
		Der an der Stelle $a\in$ Adr gespeicherte Wert $v\in$ Val ist $m(a)$.
	\end{block}
\end{frame}

\begin{frame}
	\frametitle{Speicher als Abbildung}
	\begin{exampleblock}{Beispiel}
		Gegeben sei ein Rechner mit 8\,GiB Hauptspeicher. Mit einer Adresse adressiert man ein Byte.\\
		Wie sieht die Abbildung $m$ für diesen Fall aus?
		\pause
		\begin{equation*}
			m:\{0,1\}^{64}\rightarrow \{0,1\}^8
		\end{equation*}
	\end{exampleblock}
\end{frame}



