\section{Huffman-Codierung}
\subsection{Aufgaben}
\begin{frame}
	\begin{block}{Aufgabe}
		Gegeben sei das Wort $eaebcaebacbde$.
		\begin{enumerate}
			\item Konstruiere den dazugehörigen Huffman-Baum
			\item Codiere das Wort
		\end{enumerate}
	\end{block}
\end{frame}

\begin{frame}
	\begin{block}{Aufgabe}
		\begin{itemize}
			\item Codiere das Wort $badcfehg$.
			\item Wie lang wird die Codierung?
		\end{itemize}
	\end{block}
	\begin{block}{Aufgabe}
		\begin{itemize}
			\item Es sei folgende Zeichenverteilung gegeben:
		\end{itemize}
			\begin{tabular}{ccccccccc}
				\toprule
				Zeichen&a&b&c&d&e&f&g&h\\
				\midrule
				Häufigkeit&1&2&4&8&16&32&64&128\\
				\bottomrule
			\end{tabular}
		\begin{itemize}
			\item Codiere erneut das Wort $badcfehg$.
			\item Wie lang wird dieses Mal die Codierung?
		\end{itemize}
	\end{block}
\end{frame}

\begin{frame}[plain]
	\begin{block}{Klausur-Aufgabe}
  Gegeben sei ein Wort uber dem Alphabet $A = \{a, b, c, d\}$ mit folgenden relativen Häufigkeiten:
    \begin{tabular}{ccccccccc}
      \toprule
      Zeichen&a&b&c&d&\\
      \midrule
      Häufigkeit & x & $\frac{1}{4}$ & $\frac{1}{4}$ & $\frac{1}{2} - x$\\
      \bottomrule
    \end{tabular}\\
    $0 \leq x \leq \frac{1}{4}$\pause
		\begin{enumerate}
			\item Konstruieren Sie den Huffman-Baum für $x = \frac{1}{16}$.
      \item Welche Struktur muss der Huffman-Baum haben, damit die Huffman-Codierung
            eines Wortes $w \in A^+$ mit den in der Tabelle angegebenen relativen
            Häufigkeiten echt kürzer als $2|w|$ sein kann?
      \item Für welche $x \in \mathbb{R}$ mit $0 \leq x \leq \frac{1}{4}$ werden Wörter
            mit den angegebenen relativen Häufigkeiten auf genau doppelt so
            lange Wörter über $\{0, 1\}$ abgebildet?
		\end{enumerate}
	\end{block}
\end{frame}
