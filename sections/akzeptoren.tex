\section{Endliche Automaten}
\subsection*{}
\begin{frame}
  \frametitle{Arten von Automaten}
  \begin{block}{Mealy-Automat}
    \begin{itemize}
      \item Erzeugung einer Ausgabe bei jedem Zustandsübergang
      \item Ausgabefunktion $g: Z \times X \rightarrow Y^*$
      \item Markieren der \emph{Kanten} mit $x_i|y_i$
    \end{itemize}
   \end{block}
  \pause
  \begin{block}{Moore-Automat}
    \begin{itemize}
      \item Erzeugung einer Ausgabe bei Erreichen eines Zustands
      \item Ausgabefunktion $h: Z \rightarrow Y^*$
      \item Markieren der \emph{Zustände} mit $q_i|y_i$ ($q_i$ ist Zustandsname)
    \end{itemize}
  \end{block}

  In beiden Fällen ist die Ausgabe ein Wort $y = y_0\ldots y_{n-1}$ über einem
  Ausgabealphabet $Y$.
\end{frame}

\begin{frame}
  \frametitle{Endlicher Akzeptor}
  Spezialfall eines Moore-Automaten: Ausgabealphabet $Y = \{0, 1\}$
  \begin{definition}
    $A = (Z, z_0, X, f, F)$
    \begin{description}
      \item[$Z$:] Zustandsmenge
      \item[$z_0$:] Startzustand $z_0 \in Z$
      \item[$X$:] Eingabealphabet
      \item[$f$:] $Z \times X \longrightarrow Z$, Zustandsübergangsfunktion
      \item[$F$:] Akzeptierende Zustände
    \end{description}
  \end{definition}
  \begin{itemize}
    \item Ein Wort $w\in X^*$ wird \emph{akzeptiert}, wenn gilt $f^*(z_0,w)\in F$
    \item Die von einem Akzeptor $A$ \emph{akzeptierte formale Sprache} ist $L(A)=\{w \in X^* |f^*(z_0,w)\in F\}$
   \end{itemize}
\end{frame}

